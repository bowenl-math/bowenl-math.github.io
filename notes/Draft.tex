\documentclass[11pt]{amsart}

%\usepackage{color,graphicx}
%\usepackage{mathrsfs,amsbsy}
\usepackage{amssymb}
\usepackage{CJK}
\usepackage{amsmath}
\usepackage{amsfonts}
\usepackage{graphicx}
\usepackage{amsthm}
\usepackage{enumerate}
\usepackage[mathscr]{eucal}
\usepackage{mathrsfs}
\usepackage{appendix}
\usepackage{blindtext}
\usepackage{verbatim}
\usepackage{tikz-cd}
%\usepackage[notcite,notref]{showkeys}

% showkeys  make label explicit on the paper

\makeatletter
\@namedef{subjclassname@2010}{%
  \textup{2010} Mathematics Subject Classification}
\makeatother

\numberwithin{equation}{section}

\theoremstyle{plain}
\newtheorem{theorem}{Theorem}[section]
\newtheorem{lemma}[theorem]{Lemma}
\newtheorem{proposition}[theorem]{Proposition}
\newtheorem{corollary}[theorem]{Corollary}
\newtheorem{claim}[theorem]{Claim}
\newtheorem{defn}[theorem]{Definition}

\theoremstyle{plain}
\newtheorem{thmsub}{Theorem}[subsection]
\newtheorem{lemmasub}[thmsub]{Lemma}
\newtheorem{corsub}[thmsub]{Corollary}
\newtheorem{propsub}[thmsub]{Proposition}
\newtheorem{defnsub}[thmsub]{Definition}
\newtheorem{remksub}[thmsub]{Remark}
\newtheorem{notasub}[thmsub]{Notation}
\newtheorem{exsub}[thmsub]{Example}
\newtheorem{exersub}[thmsub]{Exercise}
\newtheorem{app}[thmsub]{Application}
\numberwithin{equation}{section}

\DeclareMathOperator{\supp}{supp}
\DeclareMathOperator{\dist}{dist}
\DeclareMathOperator{\vol}{vol}
\DeclareMathOperator{\diag}{diag}
\DeclareMathOperator{\tr}{tr}

\begin{document}
\date{}

\title
{Yang-Mills Equations on Riemann Surfaces and Moduli Spaces}

\author{Bowen Liu}
\address{School of Mathematics, Shandong University\\
Jinan, 250100\\ P.R. China\\} 
\email{bowenl@mail.sdu.edu.cn}

\begin{abstract}
This article is mainly divided into four parts. In the first part, as a preparation for further theories, I studied the principal bundles and the geometry of vector bundle and principal bundle, such as covariant derivative or connection and curvature on them. I introduced the motivations of studying covariant derivative and what roles does connection of principal bundle play. In order to a better understanding of it, I studied three different views of connection. I mainly followed “Differential Geometry” written by Taubes $[2]$, and refer $[3],[4]$ for some supplements.

In the second part, I reviewed some basic definitions in geometry, such as volume form or Hodge star operator, since they're crucial in defining Yang-Mills functional. Then I gave an introduction about Yang-Mills functional and its variations.

In the third part, I introduced a fairy tale version of Kempf-Ness and an abstract version of it, since it's a quite important theorem in symplectic quotient and GIT quotient.

In the forth part, I mainly followed $[3]$, which give a neat introduction about Yang-Mills equations' applications on the stability of holomorphic vector bundles. 

\textbf{Key words:} principle bundle, connection, curvature, Yang-Mills equations, stability of holomorphic vector bundles.
\end{abstract}

\maketitle
\tableofcontents
\newpage


\section{Geometry of Principle Bundles}
\subsection{Principle bundles}
\begin{defnsub}[principle $G$-bundle]
A principle $G$-bundle is a smooth manifold $P$, with the following data:
\begin{enumerate}[$1.$]
\item A Lie group $G$ acting freely and transitively on $P$ on the right:
$$
\begin{aligned}
R_g:P\times G&\to P\\
(p,g)&\mapsto pg
\end{aligned}
$$
\item A surjective map $\pi:P\to M$ that is $G$-invariant, that is, $\pi(pg)=\pi(p),\forall p\in P,g\in G$.
\item For every point $x\in M$, there exists a neighborhood $U$ of $x$ and a diffeomorphism $\varphi:P|_U:=\pi^{-1}(U)\to U\times G$ that is $G$-equivariant, and the following diagram commutes
\begin{center}
\begin{tikzcd}
P|_U \arrow[rr, "\varphi"] \arrow[rd, "\pi"] & {} \arrow[loop, distance=2em, in=305, out=235] & U\times G \arrow[ld] \\
                                             & U                                              &                     
\end{tikzcd}
\end{center}
where $U\times G\to U$ is projection onto the first factor.
\end{enumerate}
\end{defnsub}
\begin{remksub}\normalfont
From $3$, we know that $M$ admits an open cover $\{U_{\alpha}\}$ and $G$-equivariant diffeomorphisms $\varphi_{\alpha}:\pi^{-1}U_{\alpha}\to U_{\alpha}\times G$. More explicitly, this means $\varphi_{\alpha}(p)=(\pi(p),g_{\alpha}(p))$ for some $G$-equivariant map $g_{\alpha}:\pi^{-1}U_{\alpha}\to G$ which is a fiberwise diffeomorphism. Equivariance means that 
$$
g_{\alpha}(pg)=g_{\alpha}(p)g,\quad\forall g\in G
$$
\end{remksub}
\begin{remksub}\normalfont
If we consider two local trivialization $(U_{\alpha},\varphi_{\alpha}),(U_{\beta},\varphi_{\beta})$ with non-empty intersection $U_{\alpha\beta}$, then\footnote{Here may be a little misleading, $\operatorname{Aut}(G)$ is always reserved for group automorphisms, but here $f$ is not a group automorphism, but just a diffeomorphism.} 
$$
\varphi_{\alpha}\circ\varphi_{\beta}^{-1}:U_{\beta}\times G\to U_{\alpha}\times G
$$
then for any $(m,h)\in U_{\alpha\beta}\times G$, we will $\varphi_{\alpha}\circ\varphi_{\beta}^{-1}((m,h))=(m,g_{\alpha\beta}(m)h)\in U_{\alpha\beta}\times G$, where $g_{\alpha\beta}:U_{\alpha\beta}\to\operatorname{Aut}(G)$.
 
Furthermore, since local trivialization is $G$-equivariant, then in fact
$$
g_{\alpha\beta}\in\operatorname{Aut}_G(G):=\{f:G\to G\mid f(xg)=f(x)g,\forall x,g\in G,\text{$f$ is a diffeomorphism.}\}
$$
and we have $\operatorname{Aut}_G(G)\cong G$. Indeed, $G\subseteq\operatorname{Aut}_G(G)$ automatically holds, since for any $g\in G$, we can define a map $x\mapsto gx$. Conversely, note that such $f$ is completely determined by its value at any point. If we let $g$ to denote $f(e)$, where $e\in G$ is the identity. Then $f(x)=f(ex)=gx$ for all $x\in G$, that is, $f$ is just $x\mapsto gx$. 

So not similar to case of vector bundle with fiber $V$, in which case gluing data are given by
$$
g_{\alpha\beta}:U_{\alpha\beta}\to\operatorname{GL}(V)
$$ 
The gluing data of principal bundles are given by $G$-equivariant maps
$$
g_{\alpha\beta}:U_{\alpha\beta}\to G
$$
satisfying cocycle conditions. Conversely, this determines the principal bundles uniquely as case of vector bundles.
\end{remksub}

Many principal bundles arise from vector bundles, such as frame bundles we will discuss immediately.
\begin{exsub}[frame bundle]\normalfont
Let $E\to M$ be a real vector bundle of rank $k$. For $x\in M$, let $\mathcal{B}_x$ denote the set of all bases of the fiber $E_x$, i.e. the set of linear isomorphisms $\Bbb{R}^k\to E_x$. This has a natural right action of $\operatorname{GL}_k(\Bbb{R})$ by precomposition. Then let
$$
\mathcal{B}_{\operatorname{GL}_k(\Bbb{R})}(E):=\coprod_{x\in M}\mathcal{B}_x
$$
Using local trivialization of the vector bundle $E$, we equip $\mathcal{B}_{\operatorname{GL}_k(\Bbb{R})}(E)$ with the structure of a smooth manifold such that $\pi:\mathcal{B}_{\operatorname{GL}_k(\Bbb{R})}(E)\to M$ taking $\mathcal{B}_x$ to $x$. This gives $\mathcal{B}_{\operatorname{GL}_k(\Bbb{R})}(E)$ the structure of $\operatorname{GL}_k(\Bbb{R})$-bundle, called the frame bundle of $E$.
\end{exsub}
\begin{exsub}[orthonormal frame bundle]\normalfont
Let $E\to M$ be a rank $k$ vector bundle equipped with a fiber metric, i.e. a smoothly varying inner product on the fibers $E_x$. Then the orthonormal frame bundle of $E$, denoted $\mathcal{B}_O(E)$, is the principal $O_k$-bundle where the fiber over $x\in M$ is the linear isometries $\Bbb{R}^k\to E_x$, where we use the standard inner product on $\Bbb{R}^k$ and the fiber metric restricted to $E_x$.
\end{exsub}
\begin{remksub}\normalfont
A near identical story holds for complex vector bundles: from any complex vector bundle we a principal $\operatorname{GL}_k(\Bbb{C})$-bundle of frames, and if we fix a Herimitian fiber metric, we a principle $U_k$-bundle of orthogonal frames, which is quite important in future.
\end{remksub}
\begin{defnsub}[trivial principal bundles]
A principal bundle $P$ is called trivial, if it admits a global trivialization. In other words, if there exists a diffeomorphism $\varphi:P\to M\times G, \varphi(p)=(\pi(p),g(p))$ such that $g(ph)=g(p)h,\forall h\in G$.
\end{defnsub}
\begin{remksub}\normalfont
In other words, a principal bundle is trivial if and only if it admits a global trivialization.
\end{remksub}
\begin{defnsub}[section]
A section is a smooth map $s:M\to P$ such that $\pi\circ s=\operatorname{id}$.
\end{defnsub}
\begin{remksub}\normalfont
In other words, a section is a smooth assignment to each point in the base of a point in the fiber over it. However, sections are quite rare.
\begin{propsub}
A principal bundle admits a section if and only if it is trivial.\footnote{This is in sharp contrast with vector bundles, which always admit sections.}
\end{propsub}
\begin{proof}
If $s:M\to P$ is a smooth section, we define
$$
\begin{aligned}
\varphi:P&\to M\times G\\
p&\mapsto (\pi(p),g(p))
\end{aligned}
$$
where $g(p)\in G$ is such that $p=s(\pi(p))g(p)$, it always exsits since the right action of $G$ is transitive on each fiber and it is unique since the action is free on each fiber.

Clearly, it's $G$-equivariant, since
$$
\varphi(ph)=(\pi(ph),g(ph))=(\pi(p),g(p)h)
$$
and the last equality holds since
$$
ph=s(\pi(ph))g(ph)=s(\pi(p))g(ph)=pg^{-1}(p)g(ph)\implies h=g^{-1}(p)g(ph)
$$
And it's easy to see $\varphi$ is a bijection, with inverse map
$$
\begin{aligned}
\varphi^{-1}:M\times G&\to P\\
(p,g)&\mapsto s(p)g
\end{aligned}
$$
The smoothness of the section and of the $G$-action on $P$ imply smoothness.
\end{proof}
However, since $P$ is locally trivial, local sections do exist. In fact, there are local sections $s_{\alpha}:U_{\alpha}\to\pi^{-1}U_{\alpha}$ canonically associated to the trivialization, defined so that for every $m\in U_{\alpha},\varphi_{\alpha}(s_{\alpha}(m))=(m,e)$, where $e\in G$ is the identity element. In other words, $g_{\alpha}\circ s_{\alpha}:U_{\alpha}\to G$ is the constant function sending every point to the identity.
\end{remksub}

\subsection{Associated vector bundles}
We have seen that for a vector bundle, we can construct its frame principal bundle or orthogonal frames bundle. As explained here, this construction has an inverse of sorts.

To set the stage, suppose that $G$ is a Lie group and $\pi:P\to M$ is a principal $G$ bundle. The construction of a vector bundle from $P$ requires an additional input, that is a representation $\rho$ of the group $G$ into $\operatorname{GL}(n,\Bbb{R})$ or $\operatorname{GL}(n,\Bbb{C})$, and we use $V$ to denote representation space.
\begin{defnsub}[associated vector bundle]
Let $P\to M$ be a principal $G$-bundle, and $\rho:G\to\operatorname{GL}(V)$ be a representation of $G$. The associated vector bundle, denoted by $P\times_GV$ is the space
$$
P\times_{G}V:=(P\times V)/G
$$
where the $G$-action on $P\times_GV$ is the diagonal action, i.e. $(p,v)\cdot g:=(pg,\rho(g^{-1})v)$.
\end{defnsub}
\begin{remksub}\normalfont
More generally, $V$ need not to be a vector space, it might be any manifold admitting a $G$ action.
\end{remksub}
The following proposition shows that why associated fiber bundle indeed gives us a fiber bundle with model fiber.
\begin{propsub}
$P\times_GV$ is a vector bundle over $M$ with model fiber $V$.
\end{propsub}
\begin{proof}
Consider the map taking an equivalence class $[p,f]$ to $\pi(p)$. To see the local structure, since we already have the local structure of principal bundle $P$, i.e. for any $x\in M$, there exists open $U_{\alpha}\ni x$ and $\varphi_{\alpha}:\pi^{-1}(U_{\alpha})\to U_{\alpha}\times G$. Now we define the local trivialization of $P\times_GV$ as 
$$
\begin{aligned}
\varphi^V_{\alpha}:(P\times_GV)|_{U_{\alpha}}&\to U_{\alpha}\times V\\
(p,v)&\mapsto(\pi(p),\rho(g_{\alpha}(p))v)
\end{aligned}
$$ 
First note that this is well-defined, since
$$
(pg,\rho(g^{-1})v)\mapsto(\pi(pg),\rho(g_{\alpha}(pg))\rho(g^{-1})v)=(\pi(p),\rho(g_{\alpha}(p)gg^{-1})v)=(\pi(p),\rho(g_{\alpha}(p))v)
$$
And this map is one to one, and invertible, its inverse sends $(m,v)\in U_{\alpha}\times V$ to the equivalence class of $(\varphi_{\alpha}^{-1}(m,e),v)$. Directly check as follows
$$
\begin{aligned}
\varphi_{\alpha}^V(\varphi^{-1}_{\alpha}(m,e))&=(m,\rho(e)v)\\
&=(m,v)
\end{aligned}
$$
since $\pi(\varphi_{\alpha}^{-1}(m,e))=m$ and $g_{\alpha}(\varphi_{\alpha}^{-1}(m,e))=e$.
\end{proof}
\begin{remksub}\normalfont
Though we've find the local trivialization of $P\times_GV$, it's also necessary to see what does the transition functions look like.

Let $U_{\alpha},U_{\beta}$ be open sets with non-empty intersection $U_{\alpha\beta}$, and $\varphi_{\alpha},\varphi_{\beta}$ be local trivializations of principal bundles, with transition functions 
$$
\begin{aligned}
\varphi_{\alpha}\circ\varphi^{-1}_{\beta}:U_{\alpha\beta}\times G&\to U_{\alpha\beta}\times G\\
(m,h)&\mapsto (x,g_{\alpha\beta}(m)h)
\end{aligned}
$$
then we can compute the transition functions of associated vector bundles as follows
$$
\begin{aligned}
\varphi^V_{\alpha}\circ (\varphi^V_{\beta})^{-1}: U_{\alpha\beta}\times V&\to U_{\alpha\beta}\times V\\
(m,v)&\mapsto (m, \rho(g_{\alpha\beta}(m))v)
\end{aligned}
$$
\end{remksub}
\begin{exsub}\normalfont
If $E\to M$ is a given vector bundle, with fiber $V=\Bbb{R}^n$, then we will have its frame principal $G$ bundle $P$ with $G=\operatorname{GL}(n,\Bbb{R})$. Let $\rho$ be the defining representation of $G$ on $V$, i.e. the representation $\rho:G\to\operatorname{GL}(n,\Bbb{R})$ is identity map. Then we claim that $P\times_GV$ is canonically isomorphic to $E$, since from Remark $1.2.3$ we can directly see that they have the same transition functions.
\end{exsub}
\begin{remksub}\normalfont
The above example shows that we can recover a vector bundle from its frame bundle using associated vector bundle. But we need to use defining representation, so that what “of sorts" means, i.e. principal bundles indeed encode more information.
\end{remksub}


\begin{exsub}\normalfont
There are two important examples of associated bundles that we will need to discuss the Yang-Mills equations.
\begin{enumerate}[$1.$]
\item The bundle $\operatorname{Ad}P:=P\times_GG$, where $G$ acts on $G$ by conjugation.
\item The bundle $\operatorname{ad}P:=P\times_G\mathfrak{g}$, where the action is the adjoint action. This bundle is sometimes denoted by $\mathfrak{g}_P$.
\end{enumerate}
\end{exsub}
\begin{remksub}\normalfont
Let's recall that's adjoint action, and fix some notations. For any $g\in G$. consider the conjugation $\operatorname{Ad}_g:G\to G$, defined by $h\mapsto ghg^{-1}$ and its differential. That's
$$
\begin{aligned}
\operatorname{ad}_g:\mathfrak{g}&\to\mathfrak{g}\\
X&\mapsto\left.\frac{\mathrm{d}}{\mathrm{d}t}\right|_{t=0}ge^{tX}g^{-1}
\end{aligned}
$$
So we have 
$$
\begin{aligned}
\operatorname{ad}:G&\to\operatorname{GL}(\mathfrak{g})\\
g&\mapsto\operatorname{ad}_g
\end{aligned}
$$
This representation is called adjoint representation. 

Here we use the symbol $\operatorname{ad}$ to denote the adjoint representation and $\operatorname{Ad}$ to denote the conjugation of $G$, in order to in a harmony with our symbol $\operatorname{Ad}P$ and $\operatorname{ad}P$. 

However, this may lead some confusions, since in other references authors always use $\operatorname{Ad}$ to denote adjoint representation, and use $\operatorname{ad}$ to denote its differential. Here if neccessary, we use Lie bracket ${[X,Y]}$ to denote $\operatorname{ad}_XY$ appeared in other references.
\end{remksub}

Associated bundles have another nice feature, their sections have a nice interpretation in terms of $G$-equivariant maps.
\begin{propsub}
Let $E=P\times_GV$ be an associated fiber bundle. Then there is a bijective correspondence
$$
C^{\infty}(M, E) \longleftrightarrow\{G\text {-equivariant maps } P \rightarrow V\}
$$
\end{propsub}
\begin{proof}
First, if we have a $G$-equivariant map $s^P:P\to V$, that is 
$$
s^p(pg)=\rho(g^{-1})s^p(p)
$$
then the corresponding section $s$ associates to any given point $x\in M$ the equivalence class $(p,s^P(p))$, where $p$ is any point in the fiber $P_x$. Clearly the definition is independent of the choice of $p$, since
$$
(pg,s^P(pg))=(pg,\rho(g^{-1})s^P(p))=(p,s^P(p))
$$

Conversely, suppose $s$ is a section of $P\times_GV$, by definition, $s$ associates an equivalence class in $P\times_GV$ to every point of $M$. That equivalence class gives a $G$-equivariant map $P\to V$.
\end{proof}
\begin{remksub}\normalfont
In fact, this proposition is not a coincidence, and it's a quite important motivation which explains why we need principal bundles. 

If $\pi: P\to M$ is a principal $G$ bundle, and $\pi':E\to M$ is a vector bundle such that $E$ is an associated vector bundle of $P$, then if we use $\pi$ to pull $E$ back to $P$, we claim that the vector bundle $\pi^*E$ is the trivial bundle $P\times V$ over $P$.

Indeed, we define the following bundle map
$$
\begin{aligned}
P\times V&\to P\times_GV\\
(p,v)&\mapsto[p,v]
\end{aligned}
$$
Clearly, it map the fiber over $p$ isomorphically to the fiber over $\pi(p)$, thus by the universal property of pullback, we know that $\pi^*E\cong P\times V$.

That's a quite beautiful result. And there is no wonder that section $s$ of $E$ corresponds to a $G$-equivariant map $s^P:P\to V$, since that's exactly what sections of $P\times V$ look like.

\begin{exsub}\normalfont
Let $\pi:\Bbb{C}^{n+1}\backslash\{0\}\to\Bbb{P}^n$ be the canonical projection, and we can regard $\Bbb{C}^{n+1}\backslash\{0\}$ as a principal $\Bbb{C}^*$ bundle of $\Bbb{P}^n$. Consider tautological line bundle $\mathcal{O}_{\Bbb{P}^n}(-1)$ on $\Bbb{P}^n$, and consider its pullback by $\pi$. It will be a trivial line bundle over $\Bbb{C}^{n+1}\backslash\{0\}$. Indeed, by definition
$$
\pi^{*}(\mathcal{O}(-1))=\{(x, v) \in \mathbb{C}^{n} \backslash\{0\} \times \mathcal{O}(-1) \mid \pi(x)=\pi_{L}(v)\}
$$
where $\pi_L$ is the projection of line bundle $L$. Note that $\pi(x)=[x]=\pi_L(v)$ implies that $v$ is an element of $\Bbb{C}^{n+1}$ spanned by $x$. This means that the fiber of the pullback bundle $\pi^*(\mathcal{O}_{\Bbb{P}^n}(-1))$ over $x$ is exactly $\{\lambda x\mid \lambda\in\Bbb{C}\}$. This happens globally so there is an obvious vector bundle isomorphism with $\Bbb{C}^{n+1}\backslash\{0\}\times\Bbb{C}$.

This pullback is trivial is not a coincidence, since we can regard tautological line bundle as an associated vector bundle, by considering the definition representation of $\Bbb{C}^*$ on $\Bbb{C}$.
\end{exsub}
\end{remksub}
\begin{remksub}\normalfont
This proposition shows the power of principal bundles. Instead of considering sections of $P\times_GV$, we just need to consider $G$-equivariant functions from $P$ to $V$, that is sections of trivial bundle $P\times V$ with some equivariant conditions. And it makes non-trivial things into trivial one. What a powerful result.

This view has the following useful generalization: Suppose $\pi':E'\to M$ is a second vector bundle, and perhaps unrelated to $P$. A section of $(P\times_GV)\otimes E'$ appears upstairs on $P$ as a suitably $G$-equivariant section over $P$ of $(P\times V)\otimes \pi^*E'$. We omit the proof here, but we will revisit this thing later in another viewpoint. What we need now is just need an intuition to see why we define connections on principal bundle in such a form.

The cases we will encounter are taking $E'$ to be cotangent bundle $T^*M$ or its wedge products $\bigwedge^kT^*M$. We sometimes use $\Omega_M^k(P\times_GV)$ to denote $(P\times_GV)\otimes\bigwedge^kT^*M$, the generalization tells that we have the one to one correspondence between sections of $\Omega_M^k(P\times_GV)$ and sections of $(P\times V)\otimes\pi^*\bigwedge^kT^*M$ with equivariant conditions, we will call such forms obtained by pullback basic, and denote it by $\Omega_G^k(P;V)$.
\end{remksub}

\subsection{Covariant derivatives and connections}
Let $\pi:E\to M$ be a vector bundle, and use $C^{\infty}(M,E)$ to denote the vector space of smooth sections of $E$. The questions arises as to how to take the derivative of a section $s:M\to E$ in a given direction. 

It's quite natural to ask such questions. When we learn calculus, we know how to take derivative of a smooth function $f:M\to\Bbb{R}^m$, and this gives $\mathrm{d}f:TM\to\Bbb{R}^m$, or a section of $T^*M$. Furthermore, any smooth function $f:M\to\Bbb{R}^m$ can be regarded as a section of trivial vector bundle $M\times\Bbb{R}^m$, as follows
$$
x\mapsto(x,f(x))
$$
and we can also regard its derivative $\mathrm{d}f$ as a section of $T^*M\otimes M\times\Bbb{R}^m$.

Recall how we define $\mathrm{d}f$ in the model case $f:\Bbb{R}^n\to\Bbb{R}^m$. In this setting the derivative $\mathrm{d}f$ at $x$ in the direction $v$ is defined by the standard formula
$$
\mathrm{d}f(v)(x)=\lim_{t\to0}\frac{f(x+tv)-f(x)}{t}
$$

But when passing to a section $s$ of vector bundle $E$, one encounters two key issues with this definition. Firstly, since there is no linear structure on a manifold, the term $x+tv$ makes no sense on $M$. However, this issue is relatively easy to handle. Instead one takes a path $\gamma:(-1,1)\to M$ such that $\gamma(0)=x,\gamma'(0)=v$ and computes
$$
\mathrm{d}s(v)(x)=\lim_{t\to0}\frac{s(\gamma(t))-s(\gamma(0))}{t}
$$
and that's what we do in the case of functions on manifold. However, this still does not make sense in the case of vector bundles, since $f(\gamma(t))$ and $f(\gamma(0))$ are elements of distinct fibers $E_{\gamma(t)}$ and $E_{\gamma(0)}$. This means that substraction of these two terms is not naturally defined.

One naive ideal is that if we can “transport" vectors in $E_{\gamma(t)}$ “parallel" to those in $E_{\gamma(0)}$, then we can do substraction. That's one viewpoint to think connections. To be more explicit, a connection can be viewed as assigning to every differentiable path $\gamma$ a linear isomorphisms $P_t^{\gamma}:E_{\gamma(t)}\to E_x$ for all $t$, and we can define
$$
\nabla_vf=\lim_{t\to}\frac{P^{\gamma}_tf(\gamma(t))-f(\gamma(0))}{t}
$$

Another way to solve this issue is what we're going to use in the following. Since $\mathrm{d}f$ can be regarded as a section of $\Omega_M^1$, we can directly define $\nabla s$ is also a section of $\Omega_M^1(E)$. Obviously, there are some other requirements.

As explained in what follows, we will define covariant derivatives on vector bundles and conventions on principal bundles in this section. Furthermore, we will see how these things blend with each other, and see the motivations of why do we need principal bundles to some extend.

\subsubsection{Covariant derivatives}
As it's known to all that derivative satisfies the Leibniz rule, a quite meaningful rule. It says that
$$
\mathrm{d}(fg)=\mathrm{d}fg+f\mathrm{d}g
$$
However, we can regard $f$ multiply $g$ as a $C^{\infty}(M)$-module structure of $C^{\infty}(M)$. Since on the space of smooth sections of $E$, we can also give such a module structure. The action is such that a given function $f$ acts on $s\in C^{\infty}(M,E)$ to give $fs$. So if we want to define a derivative on it, it should be compatible with such structure, like Leibniz rule tells us.
\begin{defnsub}[covariant derivative]
A covariant derivative for $C^{\infty}(M,E)$ is a linear map
$$
\nabla:C^{\infty}(M,E)\to C^{\infty}(M,\Omega_M^1(E))
$$
such that 
$$
\nabla(fs)=f\nabla s+s\otimes\mathrm{d}f
$$
for all $f\in C^{\infty}(M)$.
\end{defnsub}
\begin{remksub}\normalfont
We need to make sure that there do exists such a covariant derivative. We can do it locally and use partition of unity to a global one. For a local trivialization $(U,\varphi_U)$ of $E$, we have 
$$
\varphi_U:E|_U\to U\times V
$$
then section $v$ of such trivial bundle $U\times V$ is just like $x\mapsto (x,v(x))$. Now define the covariant derivative as $\mathrm{d}v:x\mapsto(x,\mathrm{d}v|_x)$.

Then define covariant derivative for any given section $s$ of $E$ as 
$$
\nabla s=\sum_{U}\chi_U\varphi_{U}^*(\mathrm{d}(\varphi_U\circ s|_U))
$$
\end{remksub}

\subsubsection{The space of covariant derivatives}
In fact, there are lots of covariant derivatives. As we will explain next, the space of covariant derivatives is an affine modeled on $C^{\infty}(M,\operatorname{Hom}(E,\Omega_M^1(E)))$.

To see this, first note that if $\mathfrak{a}\in C^{\infty}(M,\operatorname{Hom}(E,\Omega_M^1(E)))$ and $\nabla$ is any given covariant derivative, then $\nabla+\mathfrak{a}$ is also a covariant derivative. Meanwhile, if $\nabla$ and $\nabla'$ are both covariant derivatives, then their difference $\nabla-\nabla'$, is a section of $\operatorname{Hom}(E,\Omega_M^1(E))$. Indeed, since their difference is linear over the action of $C^{\infty}(M)$, i.e.
$$
\nabla-\nabla'(fs)=(f\nabla s+s\otimes\mathrm{d}f)-(f\nabla' s+s\otimes\mathrm{d}f)=f(\nabla-\nabla')(s)
$$
and consider the following lemma
\begin{lemmasub}
Suppose that $E$ and $E'$ are two vector bundles and $\mathcal{L}$ is a linear map that takes a section of $E$ to one of $E'$. Suppose in addition that $\mathcal{L}(fs)=f\mathcal{L}(s)$ for all functions. Then there exists a unique section $L$ of $\operatorname{Hom}(E,E')$ such that $\mathcal{L}=L$.
\end{lemmasub}
\begin{proof}
The proof is quite easy. To find $L$, fix an open set $U\subset M$ where both $E$ and $E'$ has a basis of sections. Denote the basis for $E$ and $E'$ as $\{e_i\}_{1\leq i\leq d}$ and $\{e_j'\}_{1\leq j\leq d'}$, where $d$ and $d'$ denote here the respective fiber dimensions of $E$ and $E'$. Then we have functions $\{L_{ij}\}$ such that 
$$
\mathcal{L}e_i=\sum_{1\leq j\leq d'}L_{ij}e'_j
$$
This understood, the homomorphism $L$ is defined over $U$ as follows: Let $s$ denote a section of $E$ over $U$, and write $s=\sum_{i}s_ie_i$ in terms of the basis. Then $Ls$ is defined to be the section of $E'$ as 
$$
Ls=\sum_{i,j}L_{ij}s_ie_j'
$$
The identity $\mathcal{L}s=Ls$ holds by the fact that $\mathcal{L}(fs)=f\mathcal{L}(s)$ for any functions. And the homomorphism $L$ does not depend on the choice of the basis of sections by the same reason, since $\mathcal{L}$ can commute with transition functions, this is to say that any two choices give the same section of $\operatorname{Hom}(E,E')$.
\end{proof}

However, it's more traditional to view $\nabla-\nabla'$ as a section of $\Omega_M^1(\operatorname{End}(E))$ rather than $\operatorname{Hom}(E,\Omega_M^1(E))$, since they're canonically isomorphic to each other.

What was just said about the affine property has the following implication: Let $\nabla$ be a covariant derivative on sections of $E$, and $s$ be a section of $E$. Suppose $U$ is a local trivialization of $E$, that is, $\varphi_U:E|_U\to U\times V$ is an bundle isomorphism. Write $\varphi_Us$ as $x\mapsto(x,s_U(x))$ with $s_U:U\to V$, then $\varphi_U(\nabla s)$ appears as the section
$$
x\mapsto (x,(\nabla s)_U),\quad (\nabla s)_U=\mathrm{d}s_U+a_Us_U
$$
where $a_U$ is some $s$-independent section of $\Omega_U^1(\operatorname{End}(V))$.

But $U\mapsto a_U$ does not define a section over $M$ of $\Omega_M^1(\operatorname{End}(E))$. Indeed, suppose $U'$ is another trivialization, and $g_{U'U}$ is the transition function. Thus $s_{U'}=g_{U'U}s_U$. Meanwhile, $\nabla s$ is a bonafide section of $\Omega_M^1(E)$, so $(\nabla s)_{U'}=g_{U'U}(\nabla s)_U$. This requires that 
$$
\begin{aligned}
(\nabla s)_{U'}=\mathrm{d}(g_{U'U}s_U)+a_{U'}g_{U'U}s_U&=g_{U'U}(\mathrm{d}s_U+a_Us_U)
\end{aligned}
$$
that is 
$$
a_{U'}=g_{U'U}a_Ug_{U'U}^{-1}-(\mathrm{d}g_{U'U})g_{U'U}^{-1}
$$
Conversely, we can say a covariant derivative is defined by such data, that is a collection $\{a_U\}$ satisfying above equations, where $a_U$ is a section of $\Omega_U^1(\operatorname{End}(V))$.
\subsubsection{Connections on principal bundles}
This part we will define the notion of a connection on a principal bundle. This notion is of central importance in its own right. In any event, connections are used to give an alternate and very useful definition of the covariant derivative.

Suppose vector bundle $E$ is associated to some principal bundle $\pi:P\to M$, and written as $P\times_{G}V$. Such principal bundles do exist, since we already know that $E$ is associated to its frame principal bundle.

The reason we do this is from Proposition $1.2.9$, we have a one to one correspondence between sections of $E$ with $G$-equivariant maps from $P$ to $V$. It's quite easy to take derivatives of $s^P$, it's a vector of differential $1$-forms on $P$, and it defines a fiber-wise linear map $(s^P)_*:TP\to V$. 

However, it does not by itself define a covariant derivative. As what we've defined, $\nabla s\in C^{\infty}(M,\Omega_M^1(E))$. so by Remark $1.2.12$, a covariant derivative appears upstairs on $P$ is supposed to be a $G$-equivariant section over $(P\times V)\otimes\pi^*T^*M$, that is a $G$-equivariant homomorphism from $\pi^*TM$ to $V$.

To see what $(s^P)_*$ is missing, it is important to keep in mind that $TP$ has some properties that arise from the fact that $P$ is a principal bundle over $M$. In fact, we have the following exact sequence
\begin{equation}\label{1}
0\to\operatorname{ker}(\pi_*)\to TP\to\pi^*TM\to 0
\end{equation}
This exact sequence is quite important, let's make following remarks:
\begin{remksub}\normalfont
The map from $\operatorname{ker}(\pi_*)$ is clearly an inclusion. And the map from $TP$ to $\pi^*TM$ is characterized as follows
$$
\begin{aligned}
TP&\to \pi^*TM\subset P\times TM\\
v&\mapsto (p,\pi_*v)
\end{aligned}
$$
where $v$ is the tangent vector lying over $p$.
\end{remksub}
\begin{remksub}\normalfont
We can identify $\operatorname{ker}(\pi_*)$ with something quite simply, using the special property of principal bundle. Note that $\operatorname{ker}(\pi_*)$ designates the subbundle in $TP$ that is sent by $\pi_*$ to the zero sections in $TM$. That is to say that the vectors in the $\operatorname{ker}(\pi_*)$ are thoes that are tangent to the fibers of $\pi$. Thus, $\operatorname{ker}(\pi_*)$ over $P_x$ is canonically isomorphic to $T(P_x)$, and that's Lie algebra $\mathfrak{g}$ of Lie group $G$.

In fact, we have $\operatorname{ker}(\pi_*)$ is isomorphic to trivial bundle $P\times\mathfrak{g}$. Indeed, we have the following bundle isomorphism
$$
\begin{aligned}
\psi:P\times\mathfrak{g}&\to\operatorname{ker}(\pi_*)\\
(p,X)&\mapsto \left.\frac{\mathrm{d}}{\mathrm{d} t}\right|_{t=0}pe^{tX}
\end{aligned}
$$
In other words, we can define a map $\sigma$ from $\mathfrak{g}$ to $\mathfrak{X}(M)$, the set of vector fields on $M$, by assigning to each $X\in\mathfrak{g}$, the fundamental vector field $\sigma(X)$ whose value at $p$ is given by $\psi(p,X)$.
\end{remksub}
\begin{remksub}\normalfont
Since there is a $G$ action on $P$. So you may wonder if there is a $G$ action on this exact sequence? The answer is yes. The action of $G$ on $P$ can be lifted to the exact sequence $(1.1)$, as follows: 

Let $R_g:P\to P$ denote the action of $g\in G$ on $P$, given by $p\mapsto pg$. This action lifts up to $TP$ so as the pushout $(R_g)_*:TP\to TP$. And this action descends to $\operatorname{ker}(\pi_*)$. Indeed, if $v\in TP$ is in $\operatorname{ker}(\pi_*)$, then so is $(R_g)_*v$. We can check directly as follows, take $v\in\operatorname{ker}(\pi_*)$
$$
\begin{aligned}
\pi_*((R_g)_*v)&=(\pi\circ R_g)_*(v)\\
&=\pi_*(v)\\
&=0
\end{aligned}
$$
Here we use the fact that $\pi\circ R_g=\pi$. 

The action lift of $R_g$ to $\pi^*TM$ is defined by viewing the latter in the manner described above as a subset of $P\times TM$. Viewed in this way, $R_g$ act so as to send a pair $(p,v)\in P\times TM$ to the pair $(pg,v)$. Clearly $(pg,v)\in\pi^*TM$, since $\pi(pg)=\pi(p)$.

Furthermore, the exact sequence $(1.1)$ is equivariant with respect to the lifts. Indeed, this automatically holds for inclusion map from $\operatorname{ker}(\pi_*)$ to $TP$. And it holds for the map from $TP$ to $\pi^*TM$, since for $v\in TP$ we have $(R_g)_*v$ is sent to $(pg,\pi_*(R_g)_*v)$, that is exactly $(pg,\pi_*v)$, since $\pi\circ R_g=\pi$.

By Remark $1.3.6$, we know that we can view $\operatorname{ker}(\pi_*)$ as the trivial bundle $P\times\mathfrak{g}$, so it's know how $G$ acts on it if we do this identification. That is we need to choose a $G$ action on $\mathfrak{g}$ properly such that the isomorphism $\psi$ is $G$-equivariant. 

This requires $G$ acts on $\mathfrak{g}$ by adjoint representation. In other words, $G$ acts on $X$ by sending $X$ to $g^{-1}Xg$ for $X\in\mathfrak{g}$.
Indeed, we compute as follows
$$
\begin{aligned}
(R_g)_*\psi(p,X) &=(R_g)_*\left(\left.\frac{d}{d t}\right|_{t=0} p\exp (t X)\right) \\
&=\left.\frac{d}{d t}\right|_{t=0} p\exp (t X) g \\
&=\left.\frac{d}{d t}\right|_{t=0}(pg)\left(g^{-1} \exp (t X) g\right) \\
&=\psi(pg,g^{-1}Xg)
\end{aligned}
$$
\end{remksub}
So, as we have seen, there is a canonical isomorphism of $\operatorname{ker}(\pi_*)$ with $P\times\mathfrak{g}$. We call $\operatorname{ker}(\pi_*)$ vertical subbundle of $TP$, and sometimes denote it by $V$. But in the absence of any extra structure, there is no natural complement to $\operatorname{ker}(\pi_*)$.

A connection on principal bundle $P$ is neither more nor less than a $G$-equivariant splitting of the exact sequence $(1.1)$. Since $(\nabla s)^P$ is supposed to be a $G$-equivariant fiber-wise linear map $(\nabla s)^P:\pi^*TM\to V$. So if the above sequence splits, then we have
$$
TP\cong\pi^*TM\oplus\operatorname{ker}(\pi_*)
$$
By restricting $(s^P)_*$ to $\pi^*TM$, we will define a covariant derivative of $s$. This explains why we need to define connections on principal bundle as follows.

\begin{defnsub}[connection]
A connection $A$ is a linear map
$$
A:TP\to\operatorname{ker}(\pi_*)
$$
that equals the identity on the kernel of $\pi_*$, and is $G$-equivariant with respect to the $G$-action on $TP$. 
\end{defnsub}
\begin{notasub}\normalfont
We use $\mathscr{A}(P)$ to denote the set of all connections on $P$.
\end{notasub}
\begin{remksub}[horizontal distribution viewpoint]\normalfont
In other words, instead of defining a connection to be a linear projection, we can define it as a choice of horizontal distribution $H\subset TP$ such that
$$
TP=H\oplus V
$$
Since a projection is determined by its kernel. Furthermore, it satisfies some $G$-equivariant condition, that is $(R_g)_*H_p=H_{pg}$.
\end{remksub}
\begin{remksub}[$\mathfrak{g}$-valued $1$-form viewpoint]\normalfont
Now, we give another description of connections. It is often more convenient for computations to rephrase a connection in terms of $\mathfrak{g}$-valued forms.

Since we have isomorphism $\psi:P\times\mathfrak{g}\to \operatorname{ker}(\pi_*)$, then for a connection $A$ we defined in a horizontal distribution viewpoint, it can be identified with
$$
\omega:TP\to P\times\mathfrak{g}
$$
where $\omega=\psi\circ A$. However, $P\times\mathfrak{g}$ is a trivial bundle, thus we can write $\omega$ fiberwise as follows
$$
\omega:TP\to \mathfrak{g}
$$
that is, a $1$-form valued $\mathfrak{g}$. Furthermore, it satisfies
\begin{enumerate}[$1.$]
\item $\omega(\psi(p,X))=X,\forall X\in\mathfrak{g}$.
\item If $g\in G$, then $(R_g)^*\omega=g^{-1}\omega g$. In other words, $(R_g)^*\omega=\operatorname{ad}_{g^{-1}}\circ\omega$.
\end{enumerate}

Viewed this way, a connection is a section of $T^*P\otimes P\times\mathfrak{g}$ with certain additional properties that concern the $G$-action on $P$ and the sequence $(1.1)$. That is, we can regard connections as a $\mathfrak{g}$-valued differential $1$-form, and denote it by $\Omega_P^1(\mathfrak{g})$.

Furthermore, these two different views blend with each other. Given a horizontal distribution $H\subset TP$, we can define its connection form by how does it acting on a vector field $v$ as follows
$$
\omega(v)=
\begin{cases}
X,\quad v=\sigma(X)\\
0,\quad v\in H
\end{cases}
$$
Conversely, if we have a connection form $\omega$, then we define $H=\operatorname{ker}(\omega)$.
\end{remksub}

\begin{exsub}\normalfont
Now let's show a concrete example of connections on trivial principal $G$-bundle $M\times G$. In this case, we regard $G$ as a Lie subgroup of $\operatorname{GL}(n,\Bbb{R})$\footnote{In fact, every compact Lie group admits a faithful finite-dimensional representation.}, we use $g:G\hookrightarrow\operatorname{GL}(n,\Bbb{R})$ to denote this inclusion. We first define a $\mathfrak{g}$-valued form on $G$ as follows\footnote{Immediately we will see that's exactly Maurer-Cartan form.}
$$
\omega=g^{-1}\mathrm{d}g
$$
Let's check it's indeed $\mathfrak{g}$-valued: For thoes vectors that are tangent to $G$, for $g\in G$ and $X\in T_{g}G$, we have
$$
\begin{aligned}
\omega(X)&=g^{-1}\mathrm{d}g(X)\\
&=g^{-1}\left.\frac{\mathrm{d}}{\mathrm{d}t}\right|_{t=0}ge^{tX}\\
&=\left.\frac{\mathrm{d}}{\mathrm{d}t}\right|_{t=0}e^{tX}\\
&=X\in\mathfrak{g}
\end{aligned}
$$
Then we define a connection $1$-form on $M\times G$ by using the projection $M\times G\to G$ to pullback $\omega$, and still use $\omega$ to denote this form.

Clearly $\omega$ annihilates vectors that are tangent to the $M$ factor of $M\times G$. Furthermore, it's $G$-equivariant, since for any $a\in G$, we have
$$
\begin{aligned}
(R_a)^*\omega&=(ga)^{-1}\mathrm{d}(ga)\\
&=a^{-1}(g\mathrm{d}g)a\\
&=a^{-1}\omega a\\
&=\operatorname{ad}_{a^{-1}}\omega
\end{aligned}
$$
As desired.

This understood, the corresponding horizontal distribution $H$ is precisely the tangents to the $M$ factor, this factor $TM$ in the obvious splitting of $T(M\times G)=TM\oplus TG$.
\end{exsub}

In fact, this tautological connection comes from the following form in Lie algebra.
\begin{remksub}[Maurer-Cartan form]\normalfont
The Maurer-Cartan form is a $\mathfrak{g}$-valued $1$-form $\theta$ on $G$ defined by
$$
\theta_g=(L_{g^{-1}})_*:T_gG\to T_eG=\mathfrak{g}
$$
If $X$ is a left-invariant vector field, that is 
$$
X(g)=(L_g)_*X(e)
$$
where $X(g)$ means the value of $X$ at $g$. In this case, we have
$$
\theta_g(X(g))=(L_g)_*^{-1}X(g)=(L_g)_*^{-1}(L_g)_*X(e)=X(e)
$$
which is constant. Now if $X$ and $Y$ are left-invariant vector fields, then it immediately that $\theta$ satisfies the structure equation
$$
\mathrm{d}\theta(X,Y)=-\theta([X,Y])
$$
since $X(\theta(Y))=Y(\theta(Y))=0$. 

But the left-invariant fields span the tangent space at any point.(to do this we just need to pushforward a basis of $T_eG$ under diffeomorphism.) So the equation is true for any pair of vector fields $X$ and $Y$. This is well known as Maurer-Cartan equation. Now we introduce the bracket of Lie algebra-valued forms, to give another form of Maurer-Cartan equation which we will often use later.

Choose a basis of $TG$ consisting of left-invariant vector fields $\{E_i\}$, and use $\{\theta^i\}$ to denote its dual basis. Then $\{E_i(e)\}$ is a basis of Lie algebra,  Then we can write $\alpha=\sum_iE_i(e)\otimes\alpha^i$ and $\beta=\sum_jE_j(e)\otimes\beta^j$. Define the bracket of Lie algebra-valued forms ${[\alpha,\beta]}$ as 
$$
{[\alpha,\beta]}:=\sum_{i,j}[E_i(e),E_j(e)]\alpha^i\wedge\beta^j=\sum_{i,j,k}c_{ij}^kE_k(e)\alpha^i\wedge\beta^j
$$
where $c_{ij}^k$ is the structure constants of Lie algebra.

In the frame we choose, we can write $\theta=\sum_iE_i(e)\otimes\theta^i$, since by definition we have $\theta(E_i)=E_i(e)$. Note that
$$
d \theta^{i}\left(E_{j}, E_{k}\right)=-\theta^{i}\left(\left[E_{j}, E_{k}\right]\right)=-\sum_{r} c_{j k}^{r} \theta^{i}\left(E_{r}\right)=-c_{j k}^{i}=-\frac{1}{2}\left(c_{j k}^{i}-c_{k j}^{i}\right)
$$
so we have
$$
d \theta^{i}=-\frac{1}{2} \sum_{j k} c_{j k}^{i} \theta^{j} \wedge \theta^{k}
$$
Let's take exterior derivative of Maurer-Cartan form $\theta$
$$
\begin{aligned}
\mathrm{d}\theta&=\sum_iE_i(e)\otimes\mathrm{d}\theta^i\\
&=-\frac12\sum_{i,j,k}c^i_{jk}E_i(e)\otimes\theta^j\wedge\theta^k\\
&=-\frac12[\theta,\theta]
\end{aligned}
$$

Thus we have the structure equation, sometimes called the Maurer-Cartan equation for Maurer-Cartan forms
$$
\mathrm{d}\theta+\frac12[\theta,\theta]=0
$$
However, review what we have done above, the most important equation about Maurer-Cartan form is that for any two vector fields $X,Y$ we have
$$
\mathrm{d}\theta(X,Y)=-\theta([X,Y])
$$
This is called the second form of Maurer-Cartan equation, since for any two vector fields $X,Y$, by directly computing we have:
$$
[\theta,\theta](X,Y)=[\theta(X),\theta(Y)]-[\theta(Y),\theta(X)]=2[\theta(X),\theta(Y)]
$$
The above equation is an important property of the bracket of Lie algebra-valued forms. Thus the second form of Maurer-Cartan equation is equivalent to the first one. In literature, for any $\mathfrak{g}$-valued $1$-form which satisfies Maurer-Cartan equation, we call it a Maurer-Cartan form.
\begin{exsub}\normalfont
Let $G$ be a Lie subgroup of $\operatorname{GL}(n,\Bbb{R})$, and let $g:G\hookrightarrow\operatorname{GL}(n,\Bbb{R})$ be the inclusion map. Then $\omega:=g^{-1}\mathrm{d}g$ is a Maurer-Cartan form. We check step by step:

We have already seen it's $\mathfrak{g}$-valued, it suffices to show that it do satisfies the Maurer-Cartan equation. Since $g^{-1}g=\operatorname{I}$, then $(\mathrm{d}g^{-1})g+g^{-1}\mathrm{d}g=0$. Thus we have $\mathrm{d}(g^{-1})=-g^{-1}(\mathrm{d}g)g^{-1}$. Now, 
$$
\begin{aligned}
\mathrm{d}\omega&=\mathrm{d}(g^{-1})\wedge\mathrm{d}g+g^{-1}\mathrm{d}^2g\\
&=-g^{-1}(\mathrm{d}g)g^{-1}\wedge\mathrm{d}g\\
&=-g^{-1}\mathrm{d}g\wedge g^{-1}\mathrm{d}g\\
&=-\omega\wedge\omega
\end{aligned}
$$
That's it! Later we will see, that's say that the curvature of connection $g^{-1}\mathrm{d}g$ is zero, that is, a flat connection.
\end{exsub}
\begin{exsub}\normalfont
Consider $G=\operatorname{SO}(2)\subset\operatorname{GL}(2,\Bbb{R})$. We may parametrize $\operatorname{SO}(2)$ by
$$
g(\theta)=\left(\begin{array}{cc}
\cos \theta & -\sin \theta \\
\sin \theta & \cos \theta
\end{array}\right) \quad \theta \in \mathbb{R}
$$
Then directly compute we have
$$
\eta=g^{-1}\mathrm{d}g=\left(\begin{array}{cc}
0 & -d \theta \\
d \theta & 0
\end{array}\right)
$$
\end{exsub}
\end{remksub}
\begin{remksub}[guage field viewpoint]\normalfont
Now let's see connections in another pointview, which live one $M$ instead of $P$. Recall that we have local sections $s_{\alpha}:U_{\alpha}\to\pi^{-1}U_{\alpha}$ associated canonically to the trivialization of bundle, along which we can pullback the connection $1$-form $\omega$, defining in the process the following $\mathfrak{g}$-valued $1$-forms on $U_{\alpha}$.
$$
A_{\alpha}:=s_{\alpha}^*\omega\in\Omega_{U_{\alpha}}^1(\mathfrak{g})
$$
We claim that
\begin{propsub}
The restriction of the connection $1$-form $\omega$ to $\pi^{-1}U_{\alpha}$ agrees with
$$
\omega_{\alpha}=\operatorname{ad}_{g_{\alpha}^{-1}}\circ\pi^*A_{\alpha}+g_{\alpha}^*\theta
$$
where $\theta$ is Maurer-Cartan form.
\end{propsub}
\begin{proof}
If we can show $\omega_{\alpha}$ and $\omega$ agree on the image of $s_{\alpha}$, and they transform in the same way under the action of $G$, then we can our desired result. So we prove this proposition in two steps. Refer $[4]$ for more details.
\end{proof}

Now since $\omega$ is defined globally, we have that $\omega_{\alpha}=\omega_{\beta}$ on $\pi^{-1}U_{\alpha\beta}$. This allows us to relate $A_{\alpha}$ and $A_{\beta}$ on $U_{\alpha\beta}$. Indeed, we have
$$
\begin{aligned}
A_{\alpha}=s^*_{\alpha}\omega_{\alpha}&=s_{\alpha}^*\omega_{\beta}\\
&=s_{\alpha}^*(\operatorname{ad}_{g_{\beta}^{-1}}\circ\pi^*A_{\beta}+g_{\beta}^*\theta)
\end{aligned}
$$
We want to compute what is exactly $g_{\beta}\circ s_{\alpha}:U_{\alpha\beta}\stackrel{s_{\alpha}}{\longrightarrow}\pi^{-1}U_{\alpha\beta}\stackrel{g_{\beta}}{\longrightarrow}G$. Take $m\in U_{\alpha\beta}$, then by definition we have $s_{\alpha}(m)\in\pi^{-1}U_{\alpha\beta}$, such that $\varphi_{\alpha}(s_{\alpha}(m))=(m,e)\in U_{\alpha\beta}\times G$. To see $g_{\beta}(s_{\alpha}(m))$, we need to show what's $\varphi_{\beta}(s_{\alpha}(m))$. In fact, we have
$$
\begin{aligned}
\varphi_{\beta}(s_{\alpha}(m))&=\varphi_{\beta}(\varphi_{\alpha}^{-1}(m,e))\\
&=\varphi_{\beta}\circ \varphi_{\alpha}^{-1}((m,e))\\
\end{aligned}
$$
and by the definition of $g_{\beta\alpha}$, we have $g_{\beta}\circ s_{\alpha}=g_{\beta\alpha}$. Thus 
\begin{equation}
A_{\alpha}=\operatorname{ad}_{g_{\beta\alpha}}\circ A_{\beta}+g_{\beta\alpha}^*\theta
\end{equation}
\end{remksub}
In a summary, we have already three viewpoints of connections on principal bundles, they are
\begin{enumerate}[$1.$]
\item A $G$-equivariant horizontal distribution $H\subset TP$;
\item A $1$-form $\omega\in\Omega_P^1(\mathfrak{g})$ satisfying some requirements;
\item A family of $1$-forms $A_{\alpha}\in\Omega_{U_{\alpha}}^1(\mathfrak{g})$ satisfying equation $(1.2)$ on overlaps.
\end{enumerate}


\subsubsection{The space of connections}
Since we already know what's the difference of two covariant derivatives. In this section, we explore the difference of two connections $A,A'$ on $P$. Clearly $\mathfrak{a}^P:=A-A'$ annihilates $\operatorname{ker}(\pi_*)$. As a consequence, it defines a fiber-wise linear, $G$-equivariant map from $\pi^*TM$ to $\mathfrak{g}$. So  it corresponds to a $G$-equivariant section of $\mathfrak{g}_P\otimes T^*M$, where $\mathfrak{g}_P$ is the associated vector bundle of $P$ defined in Example $1.2.7$.

Conversely, suppose $\mathfrak{a}$ is a section of $\mathfrak{g}_P\otimes T^*M$, then we can give a fiber-wise linear, $G$-equivariant map $\mathfrak{a}^P:\pi^*TM\to\mathfrak{g}$, Then if $A$ is a connection then $A+\mathfrak{a}^P$ is another one.

So if $P$ has one connection, then it has infinitely many, and the space of connections over $P$, that is $\mathscr{A}(P)$ is an affine space based on $\mathfrak{g}_P\otimes T^*M$, and this is sometimes denoted by $\Omega_M^1(\mathfrak{g}_P)$.

We can use gauge field viewpoint to see this again. Let $\omega,\omega'$ be two connection $1$-forms, and use $\tau$ to denote $\omega-\omega'$. Let's see what this means on $M$.
$$
\tau_{\alpha}=s_{\alpha}^*\tau=s_{\alpha}^*(\omega-\omega')=A_{\alpha}-A_{\alpha}'
$$
Thus we have on the overlaps $U_{\alpha\beta}$ we have 
$$
\tau_{\alpha}=\operatorname{ad}_{g_{\alpha\beta}}\circ\tau_{\beta}
$$
Thus $\{\tau_{\alpha}\}$ defines a section of $\Omega_M^1(\mathfrak{g}_P)$.

\subsection{Gauge transformations}
Now we introduce the guage group of a principal $G$-bundle $P\to M$.
\begin{defnsub}[gauge group]
The gauge group is the group of $G$-automorphisms of $P$, that is, $G$-equivariant diffeomorphisms $\Phi:P\to P$ such that $\pi=\pi\circ\Phi$.
\end{defnsub}
\begin{notasub}\normalfont
We use $\mathscr{G}(P)$ to denote the gauge group of principal bundle $P$.
\end{notasub}
\begin{defnsub}[gauge transformation]
 An element of $\mathscr{G}(P)$ is called a gauge transformation.
\end{defnsub}
\begin{remksub}\normalfont
We can describe a gauge transformation in terms of trivialization. For a gauge transformation $\Phi$, we restrict it to a trivialization $\varphi_{\alpha}:\pi^{-1}(U_{\alpha})\to U_{\alpha}$. Then consider $\varphi_{\alpha}(\Phi(p))=(\pi(p),g_{\alpha}(\Phi(p)))$, which induces a map $\bar{\phi}_{\alpha}:\pi^{-1}(U_{\alpha})\to G$ by
$$
\bar{\phi}_{\alpha}(p)=g_{\alpha}(\Phi(p))g_{\alpha}(p)^{-1}
$$
By the equivariance of $g_{\alpha}$ and $\Phi$ we have 
$$
\bar{\phi}_{\alpha}(pg)=\bar{\phi}_{\alpha}(p),\quad\forall g\in G
$$
whence $\bar{\phi}_{\alpha}(p)=\phi_{\alpha}(\pi(p))$ for some $\phi_{\alpha}:U_{\alpha}\to G$. If we consider on the overlaps $m\in U_{\alpha\beta}$ with $p=\pi^{-1}(m)$. Then
$$
\begin{aligned}
\phi_{\alpha}(m) &=g_{\alpha}(\Phi(p)) g_{\alpha}(p)^{-1} \\
&=g_{\alpha}(\Phi(p)) g_{\beta}(\Phi(p))^{-1} g_{\beta}(\Phi(p)) g_{\beta}(p)^{-1} g_{\beta}(p) g_{\alpha}(p)^{-1} \\
&=g_{\alpha \beta}(m) \phi_{\beta}(m) g_{\alpha \beta}(m)^{-1} \\
&=\operatorname{Ad}_{g_{\alpha \beta}(m)} \phi_{\beta}(m)
\end{aligned}
$$
Thus $\{\phi_{\alpha}\}$ define a section of the associated fiber bundle $\operatorname{Ad}P$. We give another proof as follows.
\end{remksub}
\begin{propsub}
The group $\mathscr{G}(P)$ is isomorphic to the group of sections $\Gamma(M,\operatorname{Ad}P)$, where the group operation is pointwise multiplication.
\end{propsub}
\begin{proof}
We provide maps in both directions. Suppose we have an automorphism $\Phi:P\to P$. Since $\pi=\pi\circ\Phi$, the map $\Phi$ preserves the fibers of $\pi$. Therefore, for any $p\in P$, we have that $p$ and $\Phi(p)$ differ by the action of some $g_p\in G$. We claim that the mapping $g_{\Phi}:P\to G$ taking $p\mapsto g_p$ is equivariant with respect to the conjugation action of $G$. Indeed, what we need to do is to compute the difference between $pg$ and $\Phi(pg)$. Use the fact that $\Phi$ is $G$-equivariant and $p=\Phi(p)g_p$, we have
$$
\begin{aligned}
\Phi(pg)&=\Phi(p)g\\
&=pg_p^{-1}g\\
&=pgg^{-1}g_p^{-1}g
\end{aligned}
$$
That is
$$
pg=\Phi(pg)g^{-1}g_pg
$$
So by Proposition $1.2.9$, it do defines a section of $\operatorname{Ad}P$.

In the other direction, given a $G$-equivariant map $f:P\to G$, we a bundle automorphism $\Phi_f:P\to P$ where $\Phi_f(p)=pf(p)$. Clearly $\pi\circ\Phi_f=\pi$, since 
$$
\begin{aligned}
\pi\circ\Phi_f(p)&=\pi(pf(p))\\
&=\pi(p)
\end{aligned}
$$
And it's $G$-equivariant since
$$
\begin{aligned}
\Phi_f(pg)&=pgf(pg)\\
&=pgg^{-1}f(p)g\\
&=pf(p)g\\
&=\Phi_f(p)g
\end{aligned}
$$

The two maps we constructed are clearly inverse to each other, giving the desired correspondence.
\end{proof}

The gauge group $\mathscr{G}(P)$ acts on the space of connections naturally. We can see this from many ways.

If we regard connections as a horizontal distribution. Let $H\subset TP$ be a connection and let $\Phi:P\to P$ be a gauge transformation. Define $H^{\Phi}:=\Phi_*H$. We claim that $H^{\Phi}$ is also a connection on $P$. Indeed,
$$
\begin{aligned}
\left(R_{g}\right)_{*} H_{\Phi(p)}^{\Phi} &=\left(R_g\right)_{*} \Phi_{*} H_p \\
&=\Phi_{*}\left(R_g\right)_{*} H_p \\
&=\Phi_{*} H_{p g} \\
&=H_{\Phi(p g)}^{\Phi} \\
&=H_{\Phi(p) g}^{\Phi}
\end{aligned}
$$
that is, $H^{\Phi}$ is $G$-equivariant. Moreover, $H^{\Phi}$ is still complementary to vertical distribution $V$, since $\Phi_*$ is an isomorphism which preserves the vertical subspaces. Indeed,

\begin{lemmasub}
The fundamental vector fields $\sigma(X)$ of the $G$-action are gauge invariant.
\end{lemmasub}
\begin{proof}
We need to prove $\Phi_*\sigma(X)=\sigma(X)$ for all $\Phi\in\mathscr{G}(P)$. Compute directly by definition
$$
\begin{aligned}
\Phi_*\sigma(X)&=\Phi_*(\left.\frac{\mathrm{d}}{\mathrm{d}t}\right|_{t=0}pe^{tX})\\
&=\left.\frac{\mathrm{d}}{\mathrm{d}t}\right|_{t=0}\Phi(pe^{tX})\\
&=\left.\frac{\mathrm{d}}{\mathrm{d}t}\right|_{t=0}\Phi(p)e^{tX}\\
&=\sigma(X)
\end{aligned}
$$
\end{proof}

If we regard connections as a $\mathfrak{g}$-valued $1$-form $\omega$, what we can do is to pull it back. So we define $\omega^{\Phi}:=(\Phi^{-1})^{*}\omega$. You may wonder why we use $\Phi^{-1}$ to pullback instead of $\Phi$. The reason is that we want $\omega^{\Phi}$ is exactly the connection form of $H^{\Phi}$. Indeed, note that if $v\in\operatorname{ker}(\omega^{\Phi})\subset TP$, then
$$
\begin{aligned}
0=\omega^{\Phi}v&=(\Phi^{-1})^{*}\omega(v)=\omega(\Phi^{-1}_*v)\\
\end{aligned}
$$
So we have 
$$
\Phi^{-1}_*v\in\operatorname{ker}\omega=H
$$
This implies that $v\in \Phi_*H=H^{\Phi}$, since by Lemma $1.4.6$ we have $\Phi_*$ preserves vertical distribution, so the only thing in $V$ mapped to zero by $\Phi_*$ is zero itself.

Finally let's see the effect of gauge transformation on a gauge field. Let $m\in U_{\alpha}$ and $p\in\pi^{-1}(U_{\alpha})$. Let $A_{\alpha}$ and $A_{\alpha}^{\Phi}$ be the gauge fields on $U_{\alpha}$ which corresponds to $\omega$ and $\omega^{\Phi}$. Then they are given at $p$ by
$$
\begin{aligned}
\omega_{p} &=\mathrm{ad}_{g_{\alpha}(p)^{-1} \circ \pi^{*}} \mathrm{~A}_{\alpha}+g_{\alpha}^{*} \theta \\
\omega_{p}^{\Phi} &=\mathrm{ad}_{g_{\alpha}(p)^{-1} \circ \pi^{*}} \mathrm{~A}_{\alpha}^{\Phi}+g_{\alpha}^{*} \theta
\end{aligned}
$$
Using $\omega^{\Phi}=(\Phi^{-1})^*\omega$ we can obtain the relation between $A_{\alpha}$ and $A_{\alpha}^{\Phi}$.
$$
A^{\Phi}_{\alpha}=\operatorname{ad}_{\phi_{\alpha}}\circ(A_{\alpha}-\phi_{\alpha}^*\theta)
$$
\subsection{Curvatures}
\subsubsection{Curvatures of covariant derivatives}
For any vector bundle $\pi:E\to M$, and a section $s$ of it, we already know how to define a covariant derivative
$$
\nabla:C^{\infty}(M,E)\to C^{\infty}(M,\Omega_M^1(E))
$$
However, in the case of smooth function, we can extend exterior derivative to more general case
$$
\mathrm{d}:C^{\infty}(M,\Omega_M^k)\to C^{\infty}(M,\Omega_M^{k+1})
$$
such that $\mathrm{d}^2=0$. The reason of $\mathrm{d}^2=0$ is the following hallmark of partial derivatives: if $f$ is a smooth function on $\Bbb{R}^n$, then
$$
\frac{\partial }{\partial x_i}\frac{\partial}{\partial x_j}f=\frac{\partial }{\partial x_j}\frac{\partial}{\partial x_i}f
$$
With the property of $\mathrm{d}$, we can construct a cochain complex, that's what de Rham cohomology concern about.

So here we wonder, can we extend a covariant derivative $\nabla$, to similar things? The answer is yes. Since the soul of taking derivatives is Leibniz rule, we can extend $\nabla$ by requiring it satisfy this rule. This extension is called the exterior covariant derivative, denoted by $\mathrm{d}_{\nabla}$, and it is defined by the following rules:
\begin{enumerate}[$1.$]
\item If $\omega$ is a $k$-form and $s$ is a section of $E$, define $\mathrm{d}_{\nabla}(s\omega)=\nabla s\wedge\omega+s\mathrm{d}\omega$.
\item If $\omega_1,\omega_2$ are sections of $\Omega_M^k(E)$, then $\mathrm{d}_{\nabla}(\omega_1+\omega_2)=\mathrm{d}_{\nabla}\omega_1+\mathrm{d}_{\nabla}\omega_2$.
\end{enumerate}

Although $\mathrm{d}^2=0$, this is generally not the case for $\mathrm{d}_{\nabla}$. It is th case, $\mathrm{d}_{\nabla}^2$ defines a section of $\operatorname{End}(E)\otimes\bigwedge^2T^*M$, in other words, a section of $\Omega_M^2(\operatorname{End}(E))$. This section is sometimes denoted by $F_{\nabla}$ and it is characterized by the fact that 
$$
\mathrm{d}_{\nabla}^2m=F_{\nabla}\wedge m
$$
Indeed, suppose $\omega$ is a $k$-form and $s$ is a section of $E$, then
$$
\begin{aligned}
\mathrm{d}_{\nabla}^2(s\omega)&=\mathrm{d}_{\nabla}(\mathrm{d}_{\nabla}s\wedge\omega)+\mathrm{d}_{\nabla}(s\mathrm{d}\omega)\\
&=\mathrm{d}_{\nabla}^2s\wedge\omega-\mathrm{d}_{\nabla}s\wedge\mathrm{d}\omega+\mathrm{d}_{\nabla}s\wedge\mathrm{d}\omega+s\wedge\mathrm{d}^2\omega\\
&=\mathrm{d}_{\nabla}^2s\wedge\omega
\end{aligned}
$$
In particular, if we take $\omega=f$ and thus a function, then
$$
\mathrm{d}_{\nabla}(fs)=f\mathrm{d}_{\nabla}s
$$ 
so by Lemma $1.3.3$, we see that $\mathrm{d}_{\nabla}^2$ is given by the action of a section of $\Omega_M^2(\operatorname{End}(E))$.

This claim can be check locally, if we choose a local trivialization $U$ of $E$. And explicitly $F_{\nabla}$ on $U$ can be regarded as $\mathrm{d}A+A\wedge A$, where $A$ is a section of $\Omega_U^1(\operatorname{End}(E|_U))$.

The exterior covariant derivative of sections is defined as above in many textbooks. However, we will recover this from the connections on principal bundles viewpoint, and explain why $\mathrm{d}_{\nabla}\neq0$.
\subsubsection{Curvatures of connections}
In this section, we will define the curvature of a connection on a principal $G$-bundle and interprete it geometrically in several different ways. Along the way we will define the covariant derivative of sections of associated vector bundle, and see that how does it connect with what we have defined. 

Given a connection $H\subset TP$, we can define the horizontal projection $h:TP\to TP$ to be the projection onto the horizontal distribution along the vertical distribution. In other words, $\operatorname{im}h=H$ and $\operatorname{ker}h=V$. Since both $H$ and $V$ are invariant under the action of $G$, so is $h$.

\begin{defnsub}[curvature $2$-form]
Let $\omega\in\Omega_P^1(\mathfrak{g})$ be the connection $1$-form, then we define curvature $2$-form $\Omega:=h^*\mathrm{d}\omega$
\end{defnsub}

Later we will give a explicit formula for $\Omega$, but first let us interpret the curvature geometrically. By definition
$$
\begin{aligned}
\Omega(u, v) &=h^*\mathrm{d}\omega(u,v)\\
&=d \omega(h u, h v) \\
&=hu\cdot \omega(h v)-hv\cdot \omega(h u)-\omega([h u, h v]) \\
&=-\omega([h u, h v])
\end{aligned}
$$  
\footnote{The third equality we use the formula connecting the exterior derivative and the Lie bracket, that is
$\mathrm{d}\omega(X,Y)=X\omega(Y)-Y\omega(X)-\omega([X,Y])
$. The forth equality we use the fact that $h^*\omega=0$.}That is, $\Omega(u,v)=0$ if and only if $[hu,hv]$ is horizontal. In other words, the curvature of the connection measures the failure of integrability of the horizontal distribution $H\subset TP$.
\begin{remksub}[Frobenius integrability]\normalfont
A distribution $D \subset TP$ is said to be integrable if the Lie bracket of any two sections of $D$ lies again in $D$. The theorem of Frobenius states that a distribution is integrable if every $p \in P$ lies in a unique submanifold of $P$ whose tangent space at $p$ agrees with the subspace $D_p \subset$ $T_pP$. These submanifolds are said to foliate $P$. As we have just seen, a connection $H \subset TP$ is integrable if and only if its curvature 2 -form vanishes.

In contrast, the vertical distribution $V \subset TP$ is always integrable, since the Lie bracket of two vertical vector fields is again vertical, and Frobenius's theorem guarantees that $P$ is foliated by submanifolds whose tangent spaces are the vertical subspaces. These submanifolds are of course the fibres of $\pi: P \rightarrow M$.
\end{remksub}

For curvature $\Omega$, it satisfies the following structure equation\footnote{That's also the definition of curvature in other references.}
\begin{propsub}[structure equation]
$$
\Omega=\mathrm{d}\omega+\frac12[\omega,\omega]
$$
where $[~,~]$ is the bracket of Lie algebra-valued forms.
\end{propsub}
\begin{proof}
By definition, we need to show
$$
\mathrm{d}\omega(hu,hv)=\mathrm{d}\omega(u,v)+[\omega(u),\omega(v)]
$$
for all vector fields $u,v$, since we have 
$$
\frac12[\omega,\omega](u,v)=[\omega(u),\omega(v)]
$$

Let $u,v$ be horizontal. In this case there is nothing to prove, since $\omega(u)=\omega(v)=0$ and $hu=u,hv=v$.

Let $u,v$ be vertical. So we can take $u=\sigma(X),v=\sigma(Y)$ for some $X,Y\in\mathfrak{g}$. By directly computing
$$
\begin{aligned}
d \omega(\sigma(X), \sigma(Y))+[\omega(\sigma(X)), \omega(\sigma(Y))]&
=\sigma(X)\omega(\sigma(Y))-\sigma(Y)\omega(\sigma(X))-\omega([\sigma(X), \sigma(Y)])+[X, Y] \\
&=\sigma(X) Y-\sigma(Y) X-\omega([\sigma(X), \sigma(Y)])+[X, Y] \\
&=-\omega([\sigma(X), \sigma(Y)])+[X, Y] \\
&=-\omega(\sigma([X, Y]))+[X, Y]\\
&=-{[X,Y]}+{[X,Y]}\\
&=0
\end{aligned}
$$
as desired.

Finally, let $u$ be horizontal and $v=\sigma(X)$ be vertical, then equation becomes
$$
\mathrm{d}\omega(u,\sigma(X))=0
$$
which in turn reduces to 
$$
u\omega(\sigma(X))-\omega(u)\sigma(X)+\omega([u,\sigma(X)])=\omega([u,\sigma(X)])=0
$$
In other words, it suffices to show that the Lie bracket of a vertical and a horizontal vector field is again horizontal. Firstly we consider a special horizontal vector field $u$ obtained by lifting vector field $u'$ on $M$. Clearly $[\sigma(X),u]=0$, since $u$ is $G$-equivariant, then it's equivariant under the action of $R_{\exp{tX}}$, so
$$
\begin{aligned}
{[\sigma(X),u]}&=\left.\frac{\mathrm{d}}{\mathrm{d}t}\right|_{t=0}(R_{\exp(tX)})_*u\\
&=0
\end{aligned}
$$
Then for any local frame $u_1',\dots,u_n'$ of $TM$, we have their liftings $u_1,\dots,u_n$ is a frame of horizontal vector fields. Then for any horizontal vector field $u$, we have $u=\sum_{i=1}^nf_iu_i$, then
$$
\begin{aligned}
{[\sigma(X),u]}&=\left.\frac{\mathrm{d}}{\mathrm{d}t}\right|_{t=0}(R_{\exp(tX)})_*u\\
&=\left.\frac{\mathrm{d}}{\mathrm{d}t}\right|_{t=0}(R_{\exp(tX)})_*(\sum_{i=1}^nf_iu_i)\\
&=\sum_{i=1}^n(\sigma(X)f_i)u_i
\end{aligned}
$$
As desired.
\end{proof}

Immediately we have
\begin{propsub}[Bianchi identity]
$$
h^*\mathrm{d}\Omega=0
$$
\end{propsub}
\begin{proof}
Directly computes
$$
\begin{aligned}
h^{*} d \Omega &=h^{*} d\left(d \omega+\frac{1}{2}[\omega, \omega]\right) \\
&=h^{*}\left(\frac{1}{2}[d \omega, \omega]-\frac{1}{2}[\omega, d \omega]\right) \\
&=h^{*}[d \omega, \omega] \\
&=\left[h^{*} d \omega, h^{*} \omega\right] \\
&=0
\end{aligned}
$$
\end{proof}

Since we already see how connections transform under a gauge transformation $\Phi:P\to P$, it's natural to ask how dose curvature transform. From structure equation, it's clear that 
$$
\Omega\mapsto\Omega^{\Phi}=(\Phi^{-1})^*\Omega
$$

If we pullback $\Omega$ via the canonical sections $s_{\alpha}:U_{\alpha}\to P$, then we $F_{\alpha}=s^*_{\alpha}\Omega\in\Omega^2_{U_{\alpha}}(\mathfrak{g})$. If follows from the structure equation that 
$$
F_{\alpha}=\mathrm{d}A_{\alpha}+\frac12[A_{\alpha},A_{\alpha}]
$$
Using the relation between $A_{\alpha}$ and $A_{\beta}$, we have
$$
F_{\alpha}=\operatorname{ad}_{g_{\alpha\beta}}\circ F_{\beta}
$$
In other words, $\{F_{\alpha}\}$ define a global $2$-form $F_A\in\Omega_M^2(\mathfrak{g}_P)$. And it transformed under gauge transformation as
$$
F_{\alpha}^{\Phi}=\operatorname{ad}_{\phi_{\alpha}}\circ F_{\alpha}
$$


\subsection{Relations of connections and covariant derivatives}
Although we have seen that some relationship between connections on principal bundles and covariant derivatives on vector bundles, here we will give a more detailed explaination here, along the way we will define many definitions we will use later.

\subsubsection{Basic forms}
As a warm-up, we need to do is to have a better understanding of the relation between forms on $P$ and forms on $M$, since later we will consider the case with coefficients.

\begin{defnsub}[horizontal form]
A $k$-form $\alpha\in\Omega^k_P$ is horizontal if $h^*\alpha=\alpha$.
\end{defnsub}
\begin{defnsub}[basic form]
A $k$-form $\alpha\in\Omega^k_P$ is basic, if it's horizontal and it is $G$-invariant.
\end{defnsub}
\begin{remksub}\normalfont
It is a fact that $\alpha$ is basic if and only if $\alpha=\pi^*\overline{\alpha}$ for some $\overline{\alpha}\in\Omega^k_M$ . In other words, the set of basic $k$-forms of $P$ are exactly $\pi^*\Omega_M^k$. 

Let's check one direction for the case $k=1$ to get a feeling, other cases are quite similar. If $\alpha=\pi^*\overline{\alpha}$ for some $\overline{\alpha}\in\Omega_M^1$, then for some $v\in TP$
$$
\begin{aligned}
h^*\alpha(v)&=\alpha(hv)\\
&=\pi^*\overline{\alpha}(hv)\\
&=\overline{\alpha}(\pi_*hv)\\
&=\overline{\alpha}(\pi_*v)\\
&=\pi^*\overline{\alpha}(v)\\
&=\alpha(v)
\end{aligned}
$$
that is, $\alpha$ is horizontal. Now let's check $\alpha$ is $G$-invariant, we need to show
$$
R_g^*\alpha=\alpha
$$
Take $v\in TP$ and check directly as follows
$$
\begin{aligned}
R_g^*\alpha(v)&=\alpha((R_g)_*v)\\
&=\overline{\alpha}(\pi_*(R_g)_*v)\\
&=\overline{\alpha}((\pi\circ R_g)_*v)\\
&=\overline{\alpha}(\pi_*v)\\
&=\alpha(v)
\end{aligned}
$$
This completes the check.
\end{remksub}

The same story happens in the case of forms on $P$ taking value in a vector space $V$ admitting a representation $\rho:G\to\operatorname{GL}(V)$ of $G$. 

Let $\alpha\in\Omega_P^k(V)$. Similarly we can define horizontal form and basic form, that is $\alpha$ is horizontal if $h^*\alpha=\alpha$ and $\alpha$ is basic if it's horizontal and $G$-invariant. For $G$-invariant it means for all $g\in G$,
$$
R_g^*\alpha=\rho(g^{-1})\alpha
$$
Since there is a $G$-action on $V$, and $\alpha$ take values in $V$, so $\rho(g^{-1})\alpha$ make sense.
\begin{notasub}\normalfont
We use $\Omega_{\mathrm{G}}^{k}(P;V)$ to denote the set of basic forms on $P$ taking values in $V$, that is
$$
\Omega_{\mathrm{G}}^{k}(P;V)=\left\{\alpha\in \Omega^{k}_P(V) \mid h^{*} \alpha=\alpha \text { and } R_g^* \alpha=\rho(g^{-1})\alpha\right\}
$$
\end{notasub}

We claim that there are some things similar to Remark $1.6.3$. That is, a form with coefficients in $V$ is basic if and only if it's a form pulled back from those on $M$ with coefficients in $P\times_GV$. Thus we have the following one to one correspondence
$$
\Omega_{\mathrm{G}}^{k}(P;V) \Longleftrightarrow\Omega^{k}_M(P\times_GV)
$$


With this correspondence, we can handle with $V$-valued forms on $P$ with some conditions, instead of bundle-valued forms on $M$. This is amazing result, since forms on $P$ valued in $V$ is just a section of trivial bundle $P\times V\otimes T^*P$, but forms on $M$ valued $E$ is a section of non-trivial bundle $E\otimes T^*M$. We convert a non-trivial thing into trivial one, and the cost we pay is just some restricted conditions.

\subsubsection{The covariant derivative}
The exterior derivative $\mathrm{d}:\Omega^k_P(V)\to\Omega_P^{k+1}(V)$ defines a $V$-valued de Rham complex. However, if we want to descend $\mathrm{d}$ to basic forms, something bad happens, since basic forms do not form a subcomplex. Indeed, since $\mathrm{d}\alpha$ need not to be horizontal even if $\alpha$ is. 

If we want to basic forms to form a complex, a naive way is to project $\mathrm{d}\alpha$ onto horizontal forms, i.e.
$$
\begin{aligned}
\mathrm{d}^H:\Omega_G^k(P;V)&\to\Omega_G^{k+1}(P;V)\\
\alpha&\mapsto h^*\mathrm{d}\alpha
\end{aligned}
$$
but the price we pay is that $(\mathrm{d}^H)^2\neq0$ in general, so we no longer have a complex. As we see soon, the failure of $\mathrm{d}^H$ defining a complex is measured by the curvature.

Let's give a explicit formula of $\mathrm{d}^H$. Take $k=0$ as an example. Every section $\zeta\in\Omega^0_M(P\times_GV)$ defines an equivalent function $\overline{\zeta}\in\Omega_G^0(P;V)$ obeying $R_g^*\overline{\zeta}=\rho(g^{-1})\bar{\zeta}$. By definition, the exterior covariant derivative is given by $\mathrm{d}^H\overline{\zeta}=h^*\mathrm{d}\overline{\zeta}$. Applying this to a vector field $u=u_V+hu$, we have
$$
\left(d^{H} \bar{\zeta}\right)(u)=d \bar{\zeta}(h u)=d \bar{\zeta}\left(u-u_{V}\right)=d \bar{\zeta}(u)-u_{V}(\bar{\zeta})
$$

What we need to do is to evaluate $u_V(\overline{\zeta})$. Take $u_V=\sigma(\omega(u))$, so that 
$$
u_V(\bar{\zeta})=\sigma(\omega(u))\bar{\zeta}=\left.\frac{d}{d t}\right|_{t=0} R_{g(t)}^{*} \bar{\zeta} \quad \text { for } g(t)=e^{t \omega(u)}
$$
Since $\bar{\xi}$ is $G$-equivariant, we have
$$
u_{V} \bar{\zeta}=\left.\frac{d}{d t}\right|_{t=0} \rho\left(g(t)^{-1}\right) \circ \bar{\zeta}=-\rho(\omega(u)) \circ \bar{\zeta}
$$
that is,
$$
\mathrm{d}^H\overline{\zeta}=\mathrm{d}\zeta+\rho(\omega)\overline{\zeta}
$$
If we apply twice, we have 
$$
\begin{aligned}
\left(d^{\mathrm{H}}\right)^{2} \bar{\zeta} &=h^{*} d h^{*} d \bar{\zeta} \\
&=h^{*} d(d \bar{\zeta}+\rho(\omega) \circ \bar{\zeta}) \\
&=h^{*}(\rho(d \omega) \circ \bar{\zeta}-\rho(\omega) \wedge d \bar{\zeta}) \\
&=\rho\left(h^{*} d \omega\right) \circ \bar{\zeta} \\
&=\rho(\Omega) \circ \bar{\zeta} .
\end{aligned}
$$
So curvature measures the failure of $\mathrm{d}^H$ to be a complex.

In the following, we will change the notation and write the exterior covariant derivative on basic forms as
$$
\mathrm{d}^{\omega}:\Omega^k_G(P;V)\to\Omega^{k+1}_G(P;V)
$$
to show the dependence on the connection $1$-form, and write the one on bundle valued forms on $M$ by 
$$
\mathrm{d}_A:\Omega^k_M(P\times_GV)\to\Omega^{k+1}_M(P\times_GV)
$$
to show the dependence on the gauge field. In this notation, the Bianchi identity for the curvature can be written as $\mathrm{d}_AF_A=0$.

\subsection{Flat connections and holonomy}
\subsubsection{Flat connections}
\begin{defnsub}[flat connection]
A connection on a principal bundle is said to be flat when its curvature $2$-form is identically zero.
\end{defnsub}
\begin{exsub}\normalfont
As we have seen, the simplest example is the connection on the trivial principal bundle $M\times P$ that is defined by $\omega=g^{-1}\mathrm{d}g$.
\end{exsub}

However, there are other flat connections. 
\begin{exsub}\normalfont
Let $h:M\to G$ denote any given smooth map. This defines a principal bundle isomorphism $\varphi:M\times G\to M\times G$ that send $(x,g)$ to $(x,h(x)g)$. Then consider the pullback of $\omega$ is given by
$$
\varphi^*\omega=g^{-1}\mathrm{d}g+h^{-1}\mathrm{d}h
$$
and the curvature of $\varphi^*\omega$ is zero.
\end{exsub}

So there may be quite a lot connections on $M\times G$, let's see the case $M=S^1$ for an instructive example.
\subsubsection{Flat connections on bundles over the circle}
In this section, $M=S^1$. Assume $G$ is a connected and $\pi:P\to S^1$ is a principal $G$-bundle. Every connection on $P$ is flat, since as all $2$-forms on $S^1$ is zero.

So it's natural to ask such a question: Suppose $\omega$ is a flat connection on $P$, is there an isomorphism $\varphi:S^1\times G\to P$ such that 
$$
\varphi^*\omega=g^{-1}\mathrm{d}g
$$

To answer this question, note that in fact all principal $G$-bundles over $S^1$ are trivial one, that is we can find an isomorphism $\varphi:P\to S^1\times G$. From Proposition $1.1.11$, it suffices to show that there exists a global section on $P$. 

To find one, it's more convenient to write $S^1$ as $\Bbb{Z}/2\pi\Bbb{Z}$. Fix an open interval $U\subset S^1$ containing $0$ with a principal $G$-bundle isomorphism $\varphi:P|_U\to U\times G$. Also, fix a connection $A$ and a point $p\in P|_0$. Let $\gamma_{A,p}:[0,2\pi]\to P$ denote the horizontal lift of the path given by the projection map from $[0,2\pi]\to S^1$. Here a horizontal lift means that it's a lift such that $(\gamma_{A,p})_*(\mathrm{d}t)\in H_A$.

As $P|_{2\pi}=P|_0$, then there exists an element $h_{A,p}\in G$ such that $\gamma_{A,p}(2\pi)h_{A,p}=p$. As $G$ is connected, we have a map $h:[0,2\pi]\to G$ such that $h$ joints identity $e\in G$ with $h_{A,p}$. This understood, we have a map $\sigma:[0,2\pi]\to P$ given by $t\mapsto \gamma_{A,p}(t)h(t)$ such that $\sigma(0)=\sigma(2\pi)=p$. Thus we get a continous section of $P$. Using basic tools in differential topology, we can get a smooth section of $P$. So as desired, we show that any principal $G$-bundle over $S^1$ is trivial.

Now let's back to our question. Choose a principal $G$-bundle isomorphism $P\to S^1\times G$ and view the given connection $A$ on $P$ as a connection on $S^1\times G$. If we have done this, the connection on the product can be written as the $\mathfrak{g}$-valued form $g^{-1}\mathrm{d}g+g^{-1}ng\mathrm{d}t$, where $n:S^1\to\mathfrak{g}$. Our questions now is equivalent to the following one:

Is there a principal $G$-bundle isomorphism $\varphi:S^1\times G\to S^1\times G$ such that 
$$
\varphi^*(g^{-1}\mathrm{d}g+g^{-1}ng\mathrm{d}t)=g^{-1}\mathrm{d}g
$$

To solve this question, we use the horizontal lift of tautological path that starts at the point $(0,e)\in S^1\times G$. Write $\gamma_{A,(0,e)}(2\pi)$ as $(0,h_{A,(0,e)})$ with $h_{A,(0,e)}\in G$.

We claim that the desired $\varphi$ exists if and ony if $h_{A,(0,e)}$ is the identity $e$. Indeed, first note that $h_{g^{-1}\mathrm{d}g,(0,e)}=e$ and note that any given bundle automorphisms $\varphi:S^1\times G\to S^1\times G$ can be written as $(t,g)\mapsto(t,u(t)g)$ where $u:S^1\to G$. This understood $h_{\varphi^*A,(0,e)}=u(0)h_{A,(0,e)}u(0)^{-1}$. Thus $\varphi^*A=g^{-1}\mathrm{d}g$ if and only if $e$ is conjugate to $h_{A,(0,e)}$, and $e$ is the only element conjugate to $e$.

And such connection is constructable, thus our question has a positive answer.
\subsubsection{Flat connections on bundles over $M$}
The following theorem describes all of the flat connections on principal $G$-bundles over a given manifold $M$. 

Let $\mathscr{F}_{M,G}$ be equivalent classes of pairs $(P,A)$, where $P\to M$ is a principal $G$-bundle and $A$ is a flat connection with respect to gauge equivalence.
\begin{thmsub}[Riemann-Hilbert correspondence]
The set $\mathscr{F}_{M,G}$ is in one to one correspondence with the set $\operatorname{Hom}(\pi_1(M),G)/G$, where $G$ acts on $\operatorname{Hom}(\pi_1(M),G)$ by conjugation.
\end{thmsub}

\section{The Yang-Mills Equations}
Untill now we have imposed no conditions on $M$ or on $G$, but now things will change. To discuss the Yang-Mills equations, we will restrict to compact Lie groups $G$, and $M$ will be a oriented Riemannian manifold with metric $g$. The orientation on $M$ is given by a nowhere-vanishing $n$-form, which we will take to be the volume form of the metric.
\subsection{Some geometry}
\subsubsection{Inner product on bundle valued forms}
Briefly speaking, the solutions of Yang-Mills equations are the critical point of Yang-Mills functional defined on the space of connections $\mathscr{A}(P)$. 

However, in order to do functional analysis, we need a metric or inner product on vector bundles we are interested in, that is, differential forms with coefficients in some vector bundle.

Firstly, let's see a more general cases. We want to define an inner product on forms with values in an associated vector bundle $P\times_GV$. On each $U_{\alpha}$, view such forms as forms with values in $V$, so all we need is an inner product on $V$, since we already have a Riemannian metric $g$ on $M$, which gives us an inner product on forms.

But if we want this inner product to glue well on overlaps, we must require that it is $G$-invariant, that is, for all $g\in G,v,w\in V$,
$$
\langle\rho(g)w,\rho(g)w\rangle=\langle v,w\rangle
$$
Indeed, if $\omega\in\Omega_M^k(P\times_GV)$ is represented locally by $\omega_{\alpha}\in\Omega^k_{U_{\alpha}}(V)$. On a non-empty overlap $U_{\alpha\beta}$, we have
$$
\langle\omega_{\alpha},\omega_{\alpha}\rangle=\langle\rho(g_{\alpha\beta})\omega_{\beta},\rho(g_{\alpha\beta})\omega_{\beta}\rangle=\langle\omega_{\beta},\omega_{\beta}\rangle
$$
whence it defines a global function $\langle\omega,\omega\rangle\in C^{\infty}(M)$.

The case we are interested in is $V=\mathfrak{g}$, since curvature of a connection lies in $\Omega_M^2(\mathfrak{g}_P)$. So we what we need is an inner product on Lie algebra $\mathfrak{g}$ which is invariant under the adjoint action of $G$. Since $G$ is compact, its Lie algebra $\mathfrak{g}$ is semisimple, so the Killing form $\langle\cdot,\cdot\rangle$ is a nondegenerate inner product, we're looking for!

Thus we have an inner product on the bundle $\Omega_M^k(\mathfrak{g}_P)$, and denote it by $\langle~,~\rangle$.
\subsubsection{Hodge star operator}
In this section, we need the Hodge star operator defined on $\Omega^k_M(\mathfrak{g}_P)$, since we already have a metric on it.
 
Let's first make it clear in the case of vector space with an inner product on it.
\begin{exsub}\normalfont
Let $V$ be an oriented $n$-dimensional Euclidean vector space, with standard inner product $\langle~,~\rangle$. In other words, $\langle e_i,e_j\rangle=\delta_{ij},\forall1\leq i,j\leq n$, where $\{e_1,\dots,e_n\}$ is a basis of $V$.

Let $W=V^*$, then we have a canonical inner product on $W$ induced by the one on $V$, if we still use $\langle~,~\rangle$ to denote this inner product, we have
$$
\langle e^i,e^j\rangle=\langle e_i,e_j\rangle,\quad\forall 1\leq i,j\leq n
$$

This induces an inner product on $k$-vectors $\alpha,\beta\in\bigwedge^kW$ for $0\leq k\leq n$, by defining it on decomposition $\alpha=\alpha_1\wedge\dots\wedge\alpha_k,\beta=\beta_1\wedge\dots\wedge\beta_k$ to equal
$$
\langle\alpha, \beta\rangle=\operatorname{det}(\langle\alpha_{i}, \beta_{j}\rangle_{i, j=1}^{k})
$$
and extend it to $\bigwedge^kV$ through linearity.

Clearly, there exists a canonical volume form $\operatorname{vol}\in\bigwedge^nW\cong\Bbb{R}$, that is $\operatorname{vol}=e^1\wedge\dots\wedge e^n$, since
by definition we have
$$
\langle\operatorname{vol},\operatorname{vol}\rangle=\operatorname{det}(\langle e^i,e^j\rangle_{i,j=1}^n)=\operatorname{det}(I_n)=1
$$

From linear algebra we already know the wedge product
$$
\bigwedge^kW\times\bigwedge^{n-k}W\stackrel{\wedge}{\longrightarrow}\bigwedge^nW
$$
is non-degenerate. So for any $\beta\in\bigwedge^kW$, we can define $*\beta\in\bigwedge^{n-k}W$ such that 
$$
\alpha\wedge*\beta=\langle\alpha,\beta\rangle\operatorname{vol},\quad \forall\alpha\in\bigwedge^kW
$$
where $\langle~,~\rangle$ is the inner product induced from $W$, and the one on $W$ is induced from $V$.

Clearly we have $*1=\operatorname{vol}$ and $*\operatorname{vol}=1$. Indeed, by definition, take $1,\alpha\in\bigwedge^0W\cong\Bbb{R}$, then
$$
\alpha*1=\alpha\wedge*1=\alpha\operatorname{vol}\implies *1=\operatorname{vol}
$$
Similar for $*\operatorname{vol}$. Furthermore,
$$
*e^i=(-1)^{i-1}e^1\wedge\dots\widehat{e^i}\wedge\dots\wedge e^n
$$
To show this, we also need to back to definition. For any $\alpha\in W$, write it as $\alpha=\sum_{i=1}^na_ie^i$, then
$$
\begin{aligned}
\alpha\wedge*e^{i}&=\langle\alpha,e^i\rangle\operatorname{vol}\\
&=\langle\sum_{i=1}^na_ie^i,e^i\rangle\operatorname{vol}\\
&=a_i\operatorname{vol}\\
&=a_ie^1\wedge\dots\wedge e^n
\end{aligned}
$$
From this equation, it's easy to see what $*e^i$ is exactly.

Last but not least, 
$$
**=*^2=(-1)^{k(n-k)}\operatorname{id},\quad \text{on }\bigwedge^kW
$$
the proof of it is also a routine, we omit it here.
\end{exsub}

We can carry what we have done to any vector bundles with a metric on it, since vector bundle is just vector space moving smoothly on $M$. In particular, we have Hodge star operator on $\Omega_M^k(\mathfrak{g}_P)$.
\begin{defnsub}[Hodge star operator]
There exists 
$$
*: C^{\infty}(X,\Omega_M^k(\mathfrak{g}_P))\to C^{\infty}(X,\Omega_{M}^{n-k}(\mathfrak{g}_P))
$$
such that 
$$
\alpha\wedge*\beta=\langle\alpha,\beta\rangle\operatorname{vol},\quad \forall\alpha\in C^{\infty}(X,\Omega_M^k(\mathfrak{g}_P))
$$
\end{defnsub}

\subsection{The variational problem}
With some of the preliminary results established, we arrive at the Yang-Mills functional.
\begin{defnsub}[Yang-Mills functional]
The Yang-Mills functional is the map $L:\mathscr{A}(P)\to \Bbb{R}$ given by
$$
L(A):=\|F_A\|=\int_M\langle F_A,F_A\rangle\operatorname{vol}
$$
\end{defnsub}
\begin{remksub}\normalfont
By the definition of Hodge star operator, we may rewrite Yang-Mills functional as follows
$$
L(A)=\int_MF_A\wedge *F_A
$$
The advantages of writing Yang-Mills functional in this way is that we can use some properties of Hodge operator to simplify our notations.
\end{remksub}
\begin{remksub}\normalfont
Note that for any gauge transformation $\Phi\in\mathscr{G}(P)$, we have $L(\Phi^*A)=L(A)$. Indeed,
$$
L(\Phi^*A)=\int_M\langle\operatorname{ad}_{\phi_{\alpha}}\circ F_A,\operatorname{ad}_{\phi_{\alpha}}\circ F_A\rangle\operatorname{vol}=\int_M \langle F_A,F_A\rangle\operatorname{vol}=L(A)
$$
because of this we say that $L$ is gauge invariant.
\end{remksub}

\begin{defnsub}[Yang-Mills connection]
A Yang-Mills connection is a connection $A\in\mathscr{A}(P)$ satisfying the Yang-Mills equations, i.e. a local extremum of $L$.
\end{defnsub}
\begin{notasub}\normalfont
We use $\mathscr{A}_{YM}(P)$, or briefly $\mathscr{A}_{YM}$ to denote the set of all Yang-Mills connections.
\end{notasub}
Let's see that this Yang-Mills conditions turns into a second-order partial differential equation for $A$.

Recall that $\mathscr{A}(P)$ is an affine space modelled on $\Omega^1_M(\mathfrak{g}_P)$. This means the tangent space to $\mathscr{A}(P)$ at any point is isomorphic to $\Omega^1_M(\mathfrak{g}_P)$. Given any $A\in\mathscr{A}(P)$ and $\tau\in\Omega_M^1(\mathfrak{g}_P)$. The directional derivative of Yang-Mills functional at $A$ in the direction $\tau$ is given by
$$
\left.\frac{\mathrm{d}}{\mathrm{d}t}\right|_{t=0}L(A+t\tau)
$$
And Yang-Mills condition states that this vanishes for all $\tau$. To see what this means. From structure equation we have
$$
\begin{aligned}
F_{A+t\tau} &=\mathrm{d}(A+t \tau)+\frac{1}{2}[A+t \tau, A+t \tau] \\
&=F_{A}+t(\mathrm{d} \tau+\frac{1}{2}[A, \tau]+\frac{1}{2}[\tau, A])+\frac{1}{2} t^{2}[\tau, \tau] \\
&=F_{A}+t(\mathrm{d} \tau+[A, \tau])+\frac{1}{2} t^{2}[\tau, \tau] \\
&=F_{A}+t \mathrm{d}_{A} \tau+\frac{1}{2} t^{2}[\tau, \tau]
\end{aligned}
$$
Then consider Yang-Mills functional
$$
\begin{aligned}
L(A+t\tau)&=\int_M\langle F_{A+t\tau}, F_{A+t\tau}\rangle\operatorname{vol}\\
&=\int_M\langle F_A+\frac{t^2}{2}[\tau\wedge\tau]+t\mathrm{d}_A\tau,F_A+\frac{t^2}{2}[\tau\wedge\tau]+t\mathrm{d}_A\tau\rangle\operatorname{vol}
\end{aligned}
$$
The term linear $t$ is 
$$
\int_M\langle F_A,\mathrm{d}_A\tau\rangle+\langle\mathrm{d}_A\tau,F_A\rangle\operatorname{vol}=2\int_M\langle\mathrm{d}_A\tau,F_A\rangle\operatorname{vol}
$$
Let $\mathrm{d}^*_A=(-1)^{2n+1}*\mathrm{d}_A*$ denote the formal adjoint to $\mathrm{d}_A$. Then we have
$$
\int_M\langle\mathrm{d}_A\tau,F_A\rangle\operatorname{vol}=\int_M\langle\tau,\mathrm{d}_A^*F_A\rangle\operatorname{vol}
$$
whence the Yang-Mills condition becomes the following differential equation
$$
\mathrm{d}_A^*F_A=0
$$
Then since up to sign $\mathrm{d}^*_A=*\mathrm{d}_A*$ and $*$ is an isomorphism, we have that above equation is equivalent to $\mathrm{d}_A*F_A=0$. Thus we have the first variation as follows

\begin{propsub}[The first variation]
Let $A$ be a local extremum of $L$. Then we have
$$
\mathrm{d}_A*F_A=0
$$
\end{propsub}
\begin{remksub}\normalfont
In the case that $G=U(1)$, we have that the curvature of a connection $A$ can be identified as an element of $\Omega_M^2$. Indeed, the curvature form takes value in the bundle $\mathfrak{g}_P$, but here $G=U(1)$ is abelian, thus the adjoint action on $\mathfrak{u}(1)$ is trivial, so $\mathfrak{g}_P$ is just the trivial bundle 
$$
\mathfrak{g}_P=M\times\mathfrak{u}(1)=M\times\Bbb{R}
$$

Furthermore, $A$ is a Yang-Mills connection if and only if $F_A$ is a harmonic form, that is $\Delta F_A=0$, where $\Delta=\mathrm{d}\mathrm{d}^*+\mathrm{d}^*\mathrm{d}$. Indeed, again thanks to $U(1)$ is abelian, $\mathrm{d}_A$ can be reduced to $\mathrm{d}$. and Hodge theory says that if $F_A$ is harmonic, then
$$
\begin{aligned}
0 &=\int_{M}\langle\Delta F_A, F_A\rangle\operatorname{vol} \\
&=\int_{M}\langle\mathrm{dd}^{*} F_A, F_A\rangle+\langle\mathrm{d}^{*} \mathrm{~d} F_A, F_A\rangle\operatorname{vol} \\
&=\int_{M}|\mathrm{d}^{*} F_A|^{2}+|\mathrm{d} F_A|^{2}\operatorname{vol}
\end{aligned}
$$
Thus we have 
$$
\begin{cases}
\mathrm{d}^*F_A=0\\
\mathrm{d}F_A=0
\end{cases}
$$
That is, $A$ satisfies the Yang-Mills equations.
\end{remksub}

\begin{propsub}[The second variation]
Let $A$ be a Yang-Mills connection. Then for every $\tau\in\Omega_M^1(\mathfrak{g}_P)$, we have
$$
\left.\frac{\mathrm{d}}{\mathrm{d} t}\right|_{t=0} \mathrm{d}_{A+t \tau}^{*} F_{A+t \tau}=\mathrm{d}_{A}^{*} \mathrm{d}_{A} \tau+\star\left[\tau \wedge \star F_{A}\right]
$$
\end{propsub}
\begin{remksub}\normalfont
If we think $L$ as a Morse function on $\mathscr{A}(P)$, for a Yang-Mills connection $A$, the operator $\mathrm{d}_A^*\mathrm{d}_A\tau+*[\tau\wedge*F_A]$ can be interpreted as the Hessian of $L$ at the critical point $A$. 
\end{remksub}

\section{GIT quotient and sympletic quotient}
Note that the Yang-Mills functional is guage invariant, so if a connection $A$ solves the Yang-Mills equations, so does any gauge transformed $A^{\Phi}$. In other words, the gauge group acts on $\mathscr{A}_{YM}$. The quotient $\mathscr{A}_{YM}/\mathscr{G}$ is the space of classical solutions. In general it is infinite dimensional, and the topology of this space may be quite bad. For example it may be neither hausdorff or a smooth manifold. But adding some restrictions, we do have a good correspondence, and that's main theorem for this section.

\subsection{A Fairy Tale}
To get a picture of the action of gauge group on $\mathscr{A}(P)$, let's study a finite-dimensional analogue:

Let $V$ be a complex vector space with a Hermitian inner product $\|\dot\|$. Let $S^1\to U(V)$ be an action of the circle by unitary matrices, and let $\Bbb{C}^*\to\operatorname{GL}(V)$ be the complexification of this action.

We want to understand the quotient space $V/\Bbb{C}^*$, but this space can be quite unpleasant. Let's see an example:
\begin{exsub}\normalfont
Let $\lambda\in\Bbb{C}^*$ acting on $\Bbb{C}^2$ by $(x,y)\mapsto(\lambda^{-1}x,\lambda y)$. The orbits are 
\begin{enumerate}[$1.$]
\item the conics $xy=c,c\neq0\in\Bbb{C}$;
\item the axes $y=0,x\neq 0$ or $x=0,y\neq0$;
\item the origin.
\end{enumerate}
It's clear that the topology on the orbit space is not Hausdorff, since origin lies in the closure of axes. But note that $\Bbb{C}^2\backslash\{\text{axes}\}/\Bbb{C}^*$ is Hausdorff. Indeed, it's homeomorphic to $\Bbb{C}$. 
\end{exsub}
The problems arise from that axes are not closed orbits. More generally, if we want to form Hausdorff quotients, we just need to consider only closed orbits which are closed sets.
\begin{defnsub}[stable]
A point $v\in V$ is stable if its orbit under $\Bbb{C}^*$ is closed.
\end{defnsub}
Now let's see a criterion for whether an orbit is closed or not. 
\begin{thmsub}
A point $v\in V$ is stable if and only if the function $\|\cdot\|^2$ restricted to its orbit attains its minimum.
\end{thmsub}
We can regard this function $p_v:\Bbb{C}^*\to\Bbb{R}$, defined by $p_v(g):=\|g(v)\|^2,g\in\Bbb{C}^*$. Note that since the norm is $U(V)$-invariant, then the function $p_v$ is $S^1$-invariant and descend to a function on $\Bbb{C}^*/S^1$, given by
$$
p_v(x):=\|e^x(v)\|^2,\quad x\in\Bbb{C}^*/S^1
$$
In example we mentioned above, we have $e^x(v_1,v_2)=(e^{-x}v_1,e^{x}v_2)$, so we have
$$
p_v(x)=\|v_1\|^2e^{-2x}+\|v_2\|^2e^{2x}
$$
Take its derivative and let it equal zero
$$
\frac{\mathrm{d}p_v(x)}{\mathrm{d}x}=-2\|v_1\|^2e^{-2x}+2\|v_2\|^2e^{2x}=0
$$
we have this function take its minimum at 
$$
\frac12(\log(\|v_1\|)-\log(\|v_2\|))
$$
if both $v_1$ and $v_2$ are not zero, and at $0$ if $v=0$. Furthermore, the minimum is not attained along the two punctured axes. In fact, this example is quite representative.

Here comes the proof of this criterion.
\begin{proof}
Note that the $S^1$-action is reducible, so $V$ splits into an orthogonal direct sum $V_1\oplus\dots\oplus V_n$ of $1$-dimensional representations where $S^1$ acts on $V_m$ as $v_m\mapsto \lambda^{j_m}v_m$ for some weight $j_m$. Therefore we have
$$
p_v(x)=\sum_m\|v_m\|^2e^{2j_mx}=\sum_{k=-\infty}^{\infty}a_ke^{kx}
$$
where only finitely many $a_k\neq0$. We divide our analysis into three cases:
\begin{enumerate}[$1.$]
\item $a_{k}=0$ for all $k \neq 0$. In this case the minimum is obviously attained and the orbit is obviously closed since $j_{m}=0$ so the action fixes $v$.
\item $a_{k}=0$ for all $k<0$ (resp. $\left.k>0\right)$ and $a_{k} \neq 0$ for some $k>0$ (resp. $k<0)$. In this case the minimum is obviously not attained and the orbit is obviously not closed since $e^{x}(v)$ tends to an orbit of the first type as $x \rightarrow-\infty$ (resp. $\infty$ ).
\item There is a $k>0$ and a $k^{\prime}<0$ such that $a_{k} \neq 0$ and $a_{k^{\prime}} \neq 0$. In this case the minimum is obviously attained (just do the calculus). We will now show that this implies $v$ is stable.
\end{enumerate}

Conversely, if $v$ is not stable, then $p_v$ does not attain its minimum. Indeed, if $v$ is not stable then its orbit is not closed so there exists $w \in V$ such that $w \in \overline{\mathbb{C}^{*}(v)}$ but $w \notin \mathbb{C}^{*}(v)$, so either $w=\lim _{x \rightarrow \pm \infty} e^{x}(v)$. The corresponding limit $\lim _{x \rightarrow \pm \infty} p_{v}(x)=p_{v}(w)$ is finite and hence the $j_{m}$ are either all nonpositive or all nonnegative. Since $w \neq v$ there must be one $j_{m}$ which is nonzero. It's now easy to see that the function $p_{v}(x)$ is of type II and hence does not attain its minimum.
\end{proof}
So the above criterion says if we want to understand the space of stable points, it's neccessary to understand the critical point of $p_v$. Take derivative we have
$$
\frac{d p_{v}}{d x}=2 \sum_{m=1}^{n} j_{m}\left\|v_{m}\right\|^{2} e^{2 j_{m} x}
$$
Suppose $v$ is stable and WLOG we assume its minimum occurs at $x=0$. Therefore the orbit of a stable vector contains a zero of the function
$$
\mu=\sum_{m=1}^nj_m\|v_m\|^2:V\to\Bbb{R}
$$
So we can restate above criterion as follows, and it's a fairy tale version of Kempf-Ness.
\begin{thmsub}[Kempf-Ness]
Let $V^s$ denote the space of stable vectors under the action of $\Bbb{C}^*$. Then
$$
V^s/\Bbb{C}=\mu^{-1}(0)/S^1
$$
\end{thmsub} 

Let's define more conceptions which we will see a more abstract version later. Let $V$ be a vector space and $Q:V\otimes V\to\Bbb{R}$ a non-degenerate bilinear form. Then we can make a $1$-form $\omega$ into a vector field $X$ by defining
$$
\omega(Y)=Q(X,Y),\quad\forall Y\in TV
$$
In particular, if $f:V\to\Bbb{R}$ is a function and consider its derivative $1$-form $\mathrm{d}f$. Then it corresponds to a vector field $\operatorname{Qgrad}(f)$ by defining
$$
\mathrm{d}f(Y)=Q(\operatorname{Qgrad}(f),Y)
$$

We call $f$ the Hamiltonian generating $\operatorname{Qgrad}(f)$, and $\mu$ the moment map for the circle action.
\begin{exsub}\normalfont
Take $f(x,y)=x^2+y^2$ and $Q=\mathrm{d}x\wedge\mathrm{d}y$. Then
$$
\operatorname{Qgrad}f=-y\partial_x+x\partial_y
$$
\end{exsub}
\subsection{Kempf-Ness Theorem}
In this section, we give an abstract version of Kempf-Ness theorem. Let $X$ be a Kähler manifold, $K\subset G$ denote the maximal compact subgroup, which has the property that its complexification is isomorphic to $G$. 

Suppose that the action of $K$ on $X$ is sympletic, i.e. the action of any $k\in K$ preserves the Kähler metric on $X$. Let $\mathfrak{l}$ denote the Lie algebra of $K$. Then the infinitesimal action of $K$ is given by the Lie algebra homomorphism $\mathfrak{l}\to\mathfrak{X}(X)$ defined by $\xi\mapsto X_{\xi}$, where
$$
\left(X_{\xi}\right)_{p}:=\left.\frac{d}{d t}\right|_{t=0} p \cdot \exp (t \xi)
$$

\begin{defnsub}[Hamiltonian]
A symplectic action of $K$ on $X$ is Hamiltonian if for each $\xi\in\mathfrak{l}$, there exists a function $H_{\xi}:X\to\Bbb{R}$ such that for all $p\in X$ and $v\in T_pX$ we have
$$
\omega_p((X_{\xi})_p,v)=(\mathrm{d}H_{\xi})_p(v)
$$
and the mapping $\xi\mapsto H_{\xi}$ is $K$-equivariant with respect to the right action of $K$ on $\mathfrak{l}$ by the adjoint action and precomposition with right translation $R_k$ on $C^{\infty}(X)$. The functions $H_{\xi}$ are called Hamiltonian functions.
\end{defnsub}
\begin{defnsub}[moment map]
Suppose we have a Hamiltonian action of $K$ on $X$. A moment map for the action is a $K$-equivariant map $\mu:X\to\mathfrak{l}^*$$($where the action on $\mathfrak{l}^*$ is the coadjoint action$)$ such that for any $p\in X,v\in T_pX$ and $\xi\in\mathfrak{l}$, we have
$$
\mathrm{d}\mu_p(v)(\xi)=\omega_p((X_{\xi})_p,v)
$$
\end{defnsub}
\begin{remksub}\normalfont
Let's make coadjoint action more clear:

Let $G$ be a Lie group and $\mathfrak{g}$ be its Lie algebra. Let $\operatorname{ad}:G\to\operatorname{Aut}(\mathfrak{g})$ denote the adjoint representation of $G$. Then we can define its coadjoint representation $\operatorname{ad}^*:G\to\operatorname{Aut}(\mathfrak{g}^*)$ as 
$$
\langle \operatorname{ad}_g^*\mu, Y\rangle=\langle\mu,\operatorname{ad}_{g^{-1}}Y\rangle
$$
for $g\in G,Y\in\mathfrak{g},\mu\in\mathfrak{g}^*$.
\end{remksub}
\begin{remksub}\normalfont
One thing to note is that the Hamiltonian functions can be recovered by the moment maps. If a Hamiltonian action admits a moment map, then
$$
H_{\xi}(p)=\mu(p)(\xi)
$$
\end{remksub}

Let $\langle\cdot,\cdot\rangle$ be an inner product on $\mathfrak{l}^*$ that is invariant under the coadjoint action, and $\|\cdot\|$ be the induced norm. Since $X$ is compact, then the map $\|\mu\|^2:X\to\Bbb{R}$ attains its minimum, and WLOG we assume that the minimum value is $0$.

\begin{defnsub}[symplectic quotient]
The symplectic quotient of $X$ by $K$ is the quotient space
$$
\mu^{-1}(0)/K
$$
\end{defnsub}
\begin{remksub}\normalfont
The symplectic quotient can also be referred to as the symplectic reduction. It should be noted that the symplectic quotient depends on our choice of moment map.
\end{remksub}
\begin{thmsub}
The symplectic quotient of $X$ by $K$ admits a unique Kähler structure such that the Kähler metric on $\mu^{-1}(0)/K$ is induced by the Kähler metric on $X$.
\end{thmsub}

The relationship between the GIT quotient and the symplectic quotient is given by the Kempf-Ness theorem
\begin{thmsub}[Kempf-Ness]
Suppose a complex reductive group $G$ acts on a Kähler manifold $X$ such that the action of the maximal compact subgroup $K\subset G$ is Hamiltonian and admits a moment map $\mu:X\to\mathfrak{l}^*$. Then the $G$-orbit of any semistable point contains a unique $K$-orbit minimizing $\|\mu\|^2$. This establish a homeomorphism
$$
X_{ss}/G\longleftrightarrow \mu^{-1}(0)/K
$$
\end{thmsub}

\subsection{Episode: Fubini-Study metric}
In usually setting, $X$ is a smooth projective variety with a fixed embedding $X\hookrightarrow\Bbb{CP}^N$, the Kähler metric $\omega$ is the restriction of the Fubini-Study form, and the $G$-action is induced by a homomorphism $G\to\operatorname{GL}_{N+1}(\Bbb{C})$. So it's neccessary to show a concrete example of Kähler manifold and its Kähler metric. In this section we will focus on $\Bbb{P}^n$ and Fubini-Study metric\footnote{Trivia: Here the pronunciation of “Study" is "SHTOO-dee".} on it.

Let $\varphi_{\alpha}:U_{\alpha}\to\Bbb{C}^n$ be the canonical coordinate cahrt. For $1\leq i,j\leq n$, we define a function on $\Bbb{C}^n$ by
$$
h_{i j}(z)=\frac{(1+\sum_{k=1}^{n}|z^{k}|^{2}) \delta_{ij}-\overline{z}^{i} z^{j}}{(1+\sum_{k=1}^{n}|z^{k}|^{2})^{2}}
$$

First we claim that
\begin{propsub}
The matrix valued function $H(z)=(h_{ij}(z))_{i,j=1}^n$ is smooth and $H(z)$ is a positive definite Hermitian matrix for all $z\in\Bbb{C}$.
\end{propsub}
\begin{proof}
Smooth and Hermitian is clear. We focus on its positive definiteness: Let $\langle\cdot,\cdot\rangle$ denote the standard inner product on $\Bbb{C}^n$ and $\|\cdot\|$ the norm induced from it. For each $z,w\in\Bbb{C}^n$, we have
$$
\begin{aligned}
\langle H(z)w,w\rangle&=\sum_{i=1}^n(\sum_{j=1}^nh_{ij}(z)w^j)w^i=\sum_{i=1}^n(\sum_{j=1}^n\frac{(1+\sum_{k=1}^{n}|z^{k}|^{2}) \delta_{ij}-\overline{z}^{i} z^{j}}{(1+\sum_{k=1}^{n}|z^{k}|^{2})^{2}}w^j)w^i\\
&=\frac{(1+\|z\|^2)\|w\|^2-|\langle z,w\rangle|^2}{(1+\|z\|^2)^2}
\end{aligned}
$$
By Cauchy-Schwarz inequality,
$$
|\langle z,w\rangle|^2\leq\|z\|^2\|w\|^2
$$
Hence we find
$$
\langle H(z)w,w\rangle\ge\frac{\|w\|^2}{(1+\|z\|^2)^2}\ge0
$$
we find that $\langle H(z)w,w\rangle=0$ implies that $\|w\|^2=0$. Therefore $w=0$. This shows that $H(z)$ is positive for each $z\in\Bbb{C}^n$.
\end{proof}

Let us define a function $K$ on $\Bbb{C}^n$ by
$$
K(z)=\log(1+\sum_{k=1}^n|z^k|^2)
$$
Let us compute $\overline{\partial}K$ as follows
$$
\overline{\partial}K=\frac{\sum_{j=1}^nz^j\mathrm{d}\overline{z^j}}{1+\sum_{k=1}^n|z^k|^2}
$$
Hence we have
$$
\begin{aligned}
\partial\overline{\partial}K&=\sum_{i,j=1}^n\frac{\delta_{ij}(1+\|z\|^2)-\overline{z}^iz^j}{(1+\|z\|^2)^2}\mathrm{d}z^i\wedge\mathrm{d}\overline{z}^j\\
&=\sum_{i,j=1}^nh_{ij}(z)\mathrm{d}z^i\wedge\mathrm{d}\overline{z}^j
\end{aligned}
$$
For each $0\leq\alpha\leq n$, we define a $(1,1)$-form on $U_{\alpha}$ by
$$
\omega_{\alpha}=\sqrt{-1}\sum_{i,j=1}^n(h_{ij}\circ\varphi_{\alpha})\mathrm{d}z^i_{\alpha}\wedge\mathrm{d}\overline{z}_{\alpha}^j
$$
This implies that if we denote $K_{\alpha}=K\circ\varphi_{\alpha}$ on $U_{\alpha}$, then
$$
\omega_{\alpha}=\sqrt{-1}\partial\overline{\partial}K_{\alpha}
$$

Use the decomposition $\mathrm{d}=\partial+\overline{\partial}$, we can find the following result by computing directly
$$
\mathrm{d}\omega_{\alpha}=0
$$

This shows that $\omega_{\alpha}$ is a closed $(1,1)$-form on $U_{\alpha}$ for each $\alpha$. In fact, we can show that the $(1,1)$-form on each $U_{\alpha}$ can be glued together to obtain a closed global $(1,1)$-form $\omega$ on $\Bbb{P}^n$.

On $U_{\alpha}$, let us rewrite $K_{\alpha}$ as follows
$$
\begin{aligned}
K_{\alpha}(\xi_0:\dots:\xi_n)&=\log(1+\sum_{k=1}^n|z_{\alpha}^k(\xi_0:\dots:\xi_n)|^2)\\
&=\log(1+\sum_{k=0}^n(\frac{\xi_k}{\xi_{\alpha}})^2)\\
&=\log(\sum_{i=0}^n|\xi_i|^2)-\log|\xi_{\alpha}|^2
\end{aligned}
$$
Assume that $\alpha<\beta$, then on $U_{\alpha}\cap U_{\beta}$, we have
$$
\begin{aligned}
K_{\alpha}(\xi_0:\dots:\xi_n)-K_{\beta}(\xi_0:\dots:\xi_n)&=\log|\xi_{\beta}|^2-\log|\xi_{\alpha}|=\log|\frac{\xi_{\beta}}{\xi_{\alpha}}|^2\\
&=\log|z_{\alpha}^{\beta}(\xi_0:\dots:\xi_n)|^2
\end{aligned}
$$
In other words, on $U_{\alpha}\cap U_{\beta}$, we have
$$
K_{\alpha}-K_{\beta}=\log|z_{\alpha}^{\beta}|^2
$$
so we have
$$
\overline{\partial}(K_{\alpha}-K_{\beta})\implies \partial\overline{\partial}(K_{\alpha}-K_{\beta})=0\quad \text{on }U_{\alpha}\cap U_{\beta}
$$
Hence we can a global $(1,1)$-form $\omega$ on $\Bbb{P}^n$. Since $\omega$ is globally defined, the $2$-tensor
$$
\sum_{i,j=1}^n(h_{ij}\circ\varphi_{\alpha})\mathrm{d}z_{\alpha}^i\otimes\mathrm{d}\overline{z}_{\alpha}^j
$$
is also globally defined. Since we have already verify that $H(z)$ is a positive definite matrix on $\Bbb{C}^n$, then above equation defines a Hermitian metric on $\Bbb{P}^n$ whose associated $(1,1)$-form is $\omega$. Thus we conclude that 
\begin{thmsub}
$\Bbb{P}^n$ is a Kähler manifold.
\end{thmsub}
The Kähler metric on $\Bbb{P}^n$ is called Fubini-Study metric.

\bigskip

Let's compute a concrete case for $\Bbb{P}^1$. Let $(\xi_0,\xi_1)$ be the standard coordinate on $\Bbb{C}^2$ and $(\xi_0,\xi_1)$ be the corresponding coordinate on $\Bbb{P}^1$. Let $U,V$ be the canonical covering of $\Bbb{P}^1$. and $z,w$ be the coordinate function on $U$ and $V$. In other words,
$$
z(\xi_0:\xi_1)=\frac{\xi_1}{\xi_0},\quad w(\xi_0:\xi_1)=\frac{\xi_0}{\xi_1}
$$

Let $K_U=\log(1+|z|^2)$ on $U$ and $K_V=\log(1+|w|^2)$ on $V$. Observe that 
$$
\begin{aligned}
K_U(\xi_0:\xi_1)&=\log(|\xi_0|^2+|\xi_1|^2)-\log|\xi_0|^2\\
K_V(\xi_0:\xi_1)&=\log(|\xi_0|^2+|\xi_1|^2)-\log|\xi_1|^2
\end{aligned}
$$
Hence on $U\cap V$ we have
$$
K_U-K_V=\log|z|^2
$$
Since $|z|^2=z\overline{z}$, we have
$$
\overline{\partial}(K_U-K_V)=\frac{1}{\overline{z}}\mathrm{d}\overline{z}
$$
Then we have
$$
\partial\overline{\partial}(K_U-K_V)=\partial(\frac{1}{\overline{z}})\wedge\mathrm{d}\overline{z}=0
$$
This shows that on $U\cap V$, we indeed have
$$
\partial\overline{\partial}K_U=\partial\overline{\partial}K_V
$$
then global defined $(1,1)$-form $\omega$ is 
$$
\omega= \begin{cases}\sqrt{-1} \partial \overline{\partial} K_{U} & \text { on } U \\ \sqrt{-1} \partial \overline{\partial} K_{V} & \text { on } V\end{cases}
$$
Let's compute what does $\omega$ exactly look like. We know that $\overline{\partial}K_U=\frac{z}{1+|z|^2}\mathrm{d}\overline{z}$. And hence
$$
\begin{aligned}
\partial\overline{\partial}K_U&=\partial(\frac{z}{1+|z|^2})\wedge\mathrm{d}\overline{z}\\
&=\frac{(1+|z|^2)\partial z-z\partial(1+|z|^2)}{(1+|z|^2)^2}\wedge\mathrm{d}\overline{z}\\
&=\frac{(1+|z|^2)\mathrm{d}z-|z|^2\mathrm{d}z}{(1+|z|^2)^2}\wedge\mathrm{d}\overline{z}\\
&=\frac{1}{(1+|z|^2)^2}\mathrm{d}z\wedge\mathrm{d}\overline{z}
\end{aligned}
$$
Similarly we can compute
$$
\partial\overline{\partial}K_V=\frac{1}{(1+|w|^2)^2}\mathrm{d}\omega\wedge\mathrm{d}\overline{\omega}
$$

So we have the Kähler metric $\mathrm{d}s^2$ on $\Bbb{P}^1$ is given by
$$
d s^{2}= \begin{cases}\frac{1}{\left(1+|z|^{2}\right)^{2}} d z \otimes d \overline{z} & \text { on } U \\ \frac{1}{\left(1+|w|^{2}\right)^{2}} d w \otimes d \overline{w} & \text { on } V\end{cases}
$$

And check this metric is globally defined. On $U\cap V$, we have $\omega=\dfrac1z$ and $\overline{w}=\dfrac{1}{\overline{z}}$. Hence we have
$$
\mathrm{d}\omega\otimes\mathrm{d}\overline{\omega}=\frac{1}{z^2}\mathrm{d}z\otimes\frac{1}{\overline{z}^2}\mathrm{d}\overline{z}=\frac{1}{|z|^4}\mathrm{d}z\otimes\mathrm{d}\overline{z}
$$
then
$$
\frac{1}{(1+|z|^{2})^{2}} \mathrm{d} z \otimes \mathrm{d} \overline{z}=\frac{|z|^{4}}{(1+|z|^{2})^{2}} \frac{\mathrm{d} z \otimes \mathrm{d} \overline{z}}{|z|^{4}}=\frac{1}{(|w|^{2}+1)^{2}} \mathrm{d}w \otimes \mathrm{d} \overline{w}
$$
We prove that $\mathrm{d}s^2$ is a globally defined Hermitian metric.

\section{Holomorphic Vector Bundles and Hermitian Yang-Mills Connections}
In the second part, we have already established the foundations of Yang-Mills equations in a general stage. Explicitly, in the stage of Riemannian manifold $(M,g)$.

As we have seen, when the dimension of underlying space is one, all curvature forms are trivial, so there is nothing interesting. Thus the first “non-trivial" theory arises when our underlying space is of dimension two. 

This prototype theory merits a good deal of study due to the richness of structures naturally occurring on such manifold, such as a complex structure associated to the almost complex structure determined by the Hodge star operator $*:\Omega_M^p\to\Omega_M^{2-p}$. Furthermore, smooth Hermitian vector bundle $E$ over Riemann surface have inherent holomorphic structures due to the vacuous integrability conditions on connections on $E$, in other words, this gives a correspondence between unitary connections and holomorphic structure $\overline{\partial}_E$ on $E$. Thus the study of Yang-Mills connections on Riemann surface can be put into a complex analytic framework. 

Using such ideal, we give a description of Kempf-Ness theorem which relates symplectic quotient and GIT quotient. In this section, if the underlying space is a Riemann surface, we will see there is a parallel story for the action of gauge group $\mathscr{G}$ on the space of connections $\mathscr{A}(P)$.

 We will complexify the action of $\mathscr{G}$ and state a theorem analogous to Kempf-Ness theorem, which is known as Narasimhan-Seshadri theorem.

\begin{notasub}\normalfont
For complex manifold $X$, we use $\mathcal{A}_X^k$ to denote the space of smooth complex-valued $k$-forms, and use $\mathcal{A}_X^{p,q}$ to denote the space of smooth $(p,q)$-forms. We reserve $\Omega_X$ for holomorphic forms.
\end{notasub}

\subsection{Moment map in Yang-Mills theory}
When $X$ is a Riemann surface, the space of connections has a natural symplectic form. As we already knonw, $\mathscr{A}(P)$ is affine modelled on $\Omega_X^1(\mathfrak{g}_P)$, then we consider the following non-degenerate symplectic form
$$
Q(\alpha,\beta)=\int_X\alpha\wedge\beta,\quad \alpha,\beta\in\Omega_X^1(\mathfrak{g}_P)
$$
where this integral do make senses since the real dimension of $X$ is two.

Take $\phi\in\Omega^0_M(\mathfrak{g}_P)$, we can get a vector field on $\mathscr{A}(P)$ by the action of $\nabla$ on $\phi$, that is $V=\nabla\phi$.
\begin{lemmasub}
The function $f:\nabla\to-\int_XF_{\nabla}\wedge\phi$ is a Hamiltonian functions on $\mathscr{A}(P)$ generating $V$.
\end{lemmasub}
\begin{proof}
It suffices to check 
$$
Q(\nabla\phi,A)=\mathrm{d}f(A),\quad \forall A\in\Omega_X^1(\mathfrak{g}_P)
$$
Integration by parts we have
$$
\begin{aligned}
Q(\nabla\phi,A)&=\int_X\nabla\phi\wedge A\\
&=-\int_X\phi\wedge\nabla A\\
&=-\int_X\nabla A\wedge\phi\\
\end{aligned}
$$
Note that $F_{\nabla+\varepsilon A}=F_{\nabla}+\varepsilon\nabla A+O(\varepsilon^2)$, then
$$
\begin{aligned}
\mathrm{d}f(A)&=\lim_{\varepsilon\to 0}\frac{-\int_XF_{\nabla+\varepsilon A}\wedge\phi+\int_XF_{\nabla}\wedge\phi}{\varepsilon}\\
&=-\int_X\nabla A\wedge\phi
\end{aligned}
$$
As desired.
\end{proof}
\begin{remksub}\normalfont
In our case the Lie algebra of gauge group is $\Omega^2_X(\mathfrak{g}_P)$ and the moment map is just
$$
\nabla\mapsto -F_{\nabla}
$$
The Yang-Mills functional is just the norm of the moment map.
\end{remksub}


\subsection{Regard $\mathscr{A}(P)$ as the space of holomorphic vector bundles}
Our ultimate goal is to relate moduli spaces of holomorphic vector bundles over $X$ to Yang-Mills connections.  Firstly, we want to consider $\mathscr{A}(P)$ as a space of holomorphic vector bundles.

Recall some basic conceptions in geometry of vector bundle.
\begin{defnsub}[holomorphic vector bundle]
A holomorphic vector bundle is a complex bundle $\pi:E\to X$ such that the total space $E$ is a complex manifold and projection $\pi$ is holomorphic.
\end{defnsub}

Given a holomorphic vector bundle $E\to X$, we can find a trivialization of $E$ such that the transition functions are holomorphic. In a neighborhood $U\subset X$ such that $E|_U$ is holomorphically trivial, the smooth sections can be identified with smooth functions $U\to\Bbb{C}^n$, and the holomorphic sections can be identified with holomorphic functions $U\to\Bbb{C}^n$.

We have a local operator $\overline{\partial}$, which we can apply componentwise to a local section to an operator on smooth sections over $U$. Furthermore, since $\overline{\partial}$ annihilates holomorphic functions and the transition functions are holomorphic, we have that $\overline{\partial}$ glues to a well defined operator $\overline{\partial}_E:\mathcal{A}_X^0(E)\to\mathcal{A}_X^{0,1}(E)$. The holomorphic sections of $E$ are exactly the sections annihilated by $\overline{\partial}_E$. Furthermore, the operator $\overline{\partial}_E$ extends to operator $\overline{\partial}_E:\mathcal{A}_X^{k}(E)\to\mathcal{A}_{X}^{k+1}(E)$, and satisfies the condition $\overline{\partial}_E^2=0$, since $\overline{\partial}^2=0$. 

Suppose now we have a holomorphic vector bundle $E$ with a Hermitian metric $h$. Then we can make sense of unitary frame and there is a principal $U(n)$-bundle $P$. Consider the complex vector bundle underlying the associated unitary vector bundle, we get a complex vector bundle and a Hermitian metric, or in other words, a Hermitian vector bundle.

\begin{defnsub}[unitary connection]
Let $(E,h)$ be a smooth Hermitian vector bundle on a complex manifold $X$, then a connection $\nabla$ is called unitary, if
$$
\mathrm{d}h(s,s')=h(\nabla s,s')+h(s,\nabla s')
$$
for any two sections $s,s'$ of $E$.
\end{defnsub}
\begin{defnsub}[compatible]
Together holomorphic vector bundle $E$ with Hermitian metric $h$, we say a unitary connection is compatible with holomorphic structure if $\nabla^{0,1}=\overline{\partial}_E$.
\end{defnsub}
\begin{remksub}\normalfont
Given a holomorphic vector bundle $E\to X$, there exists a compatible unitary connection $\nabla$, which is called Chern connection of this holomorphic vector bundle. However, on the Riemann surface, the converse statements still holds. In other words,
\end{remksub}

\begin{propsub}
If $P$ is a principal $U(n)$-bundle over a Riemann surface $X$ and $\nabla$ is a unitary connection, then associated vector bundle inherits the structure of a holomorphic vector bundle over $X$ such that 
$$
\nabla^{0,1}=\overline{\partial}
$$
\end{propsub}
\begin{corsub}
We can think of unitary connections on a $U(n)$-bundle as giving the structure of a holomorphic vector bundle to the associated complex vector bundle.
\end{corsub}

Using the identification of $\mathscr{A}(P)$ and $\mathscr{C}(E)$, where $\mathscr{C}(E)$ is the set of holomorphic structure on $E$, the action of the gauge group on $\mathscr{A}(P)$ now extends to an action of the complexified gauge group $\mathscr{G}_{\Bbb{C}}$, where $\mathscr{G}_{\Bbb{C}}$ denote the group of smooth bundle automorphisms of $E$. 

The space $\mathscr{C}(E)$ has a natural action by $\mathscr{G}_{\Bbb{C}}$ by conjugation. Furthermore, the orbits under this action are exactly the isomorphism classes of holomorphic structures on $E$. This is most easily seen by characterizing an isomorphism $\varphi:E\to F$ of holomorphic vector bundles as a smooth bundle isomorphism intertwing $\overline{\partial}_E$ and $\overline{\partial}_F$. 

However, the naive  quotient $\mathscr{C}(E)/\mathscr{G}_{\Bbb{C}}$ is poorly behaved. To remedy this, as in GIT, we expect there is a notion of stability of holomorphic vector bundles such that the stable $\mathscr{G}_{\Bbb{C}}$-orbits contain a unique minimum of the Yang-Mills functional, and this is Narasimhan-Seshadri theorem.

\subsection{Stability of holomorphic vector bundles}
In this section, let's discuss some stablilties of holomorphic vector bundles.
\begin{defnsub}[slope]
The slope of a holomorphic vector bundle $E\to X$ is 
$$
\mu(E):=\frac{c_1(E)}{\operatorname{rank}(E)}
$$
where we think of $c_1(E)\in H^2(X,\Bbb{Z})$ as an integer via integration over $X$.
\end{defnsub}
\begin{remksub}\normalfont
Sometimes the integer $c_1(E)$ is also referred to as the degree of $E$. One thing to note is that the slope of a holomorphic vector bundle is independent of the holomorphic structure. Both of degree and rank are topological invariants, and only depend on the underlying $C^{\infty}$ complex vector bundle.
\end{remksub}

\begin{defnsub}[stablity of holomorphic bundles]
A holomorphic vector bundle $E\to X$ is 
\begin{enumerate}[$1.$]
\item Stable if for every holomorphic subbundle $F\subset E$, we have $\mu(F)<\mu(E)$.
\item Semistable if for every holomorphic subbundle $F\subset E$, we have $\mu(F)\leq \mu(E)$.
\item Unstable if $E$ is not semistable.
\end{enumerate}
\end{defnsub}

While the slope is a topological invariant, stability is not, since we only consider holomorphic subbundles which depend on the holomorphic structure. We also note that both the degree and rank are additive in exact sequences, which immediately give us 

\begin{propsub}
Suppose we have the short exact sequence of holomorphic bundles
$$
0\to E\to F\to G\to0
$$
Then we have
$$
\mu(F)=\frac{\operatorname{deg}(E)+\operatorname{deg}(G)}{\operatorname{rank}(E)+\operatorname{rank}(G)}
$$
\end{propsub}
\begin{corsub}
Given a short exact sequence of holomorphic bundles
$$
0\to E\to F\to G\to 0
$$
If $\mu(E)\ge\mu(F)$, then $\mu(F)\ge\mu(G)$. Likewise, if $\mu(E)\leq\mu(F)$, then $\mu(F)\leq\mu(G)$.
\end{corsub}
In other words, slopes behave monitonically in short exact sequences. The terminology comes from GIT. The main result will use is
\begin{thmsub}[The Harder-Narasimhan Filtration]
Let $E\to X$ be a holomorphic vector bundle. Then $E$ admits a canonical filtration
$$
0=E_0\subset E_1\subset\dots\subset E_n=E
$$
by holomorphic subbundles $E_i$ such that $E_i/E_{i-1}$ is semistable and 
$$
\mu(E_1/E_0)>\mu(E_2/E_1)>\dots>\mu(E_n/E_{n-1})
$$
\end{thmsub}
\begin{proof}
Sketch. The main idea is that any holomorphic vector bundle has a unique maximal semistable subbundle, which we take to be $E_1$. We then take $E_2$ to be the preimage of the maximal semistable bundle of $E_1/E_0$ under the quotient map, and continue inductively.
\end{proof}
\begin{remksub}\normalfont
The slopes $\mu_i:=\mu(E_i/E_{i-1})$ gives us $n$ rational numbers. If $k$ denotes the rank $E$, then we construct an element of $\Bbb{Q}^k$ by arranging the $\mu_i$ in order, and repeating the entry $\mu_i$ a total of $\operatorname{rank}(E_i/E_{i-1})$ times. We call this vector the Harder-Narasimhan type of $E$.
\end{remksub}
\subsection{Narasimhan-Seshadri theorem.}

We now want to relate the previous discussion to our situation. Using the identification of $\mathscr{A}(P)$ and $\mathscr{C}(E)$, we want the action of $\mathscr{G}_{\Bbb{C}}$ to play the role of the complex reductive group $G$ and the gauge group $\mathscr{G}$ to play the role of the maximal compact subgroup. Since the space $\mathscr{A}(P)$ is infinite dimensional, along with the group $\mathscr{G}_{\Bbb{C}}$ and $\mathscr{G}$, we are working in an infinite dimensional setting, but we will gloss over the analytic details and work with them formally.

Our first task is to realize $\mathscr{A}(P)$ as a “Kähler manifold". Our next task is to show the action of $\mathscr{g}$ on $\mathscr{A}(P)$ is “Hamiltonian" with respect to this Kähler structure.

To summarize, we have the following analogies:
$$
\begin{aligned}
\text{Kähler manifold}&\longleftrightarrow\mathscr{A}(P)\\
\text{Complex reductive group}&\longleftrightarrow\mathscr{G}_{\Bbb{C}}\\
\text{Maximal compact subgroup $K\subset G$}&\longleftrightarrow\mathscr{G}\\
\text{Moment map $\mu$}&\longleftrightarrow A\mapsto F_A\\
\text{Norm square of the moment map $\|\mu\|^2$}&\longleftrightarrow L
\end{aligned}
$$

The last piece is something analogous to the Kempf-Ness theorem
\begin{thmsub}[Narasimhan-Seshadri]
Let $\mathscr{A}_s(P)\subset\mathscr{A}(P)$ denote the subspace of connections that are absolute minima for the Yang-Mills functional, and correspond to irreducible representations $\Gamma_{\Bbb{R}}\to U_n$. Let $\mathscr{C}_s(E)$ denote the subspace of stable holomorphic structures on $E$. The isomorphism classes of holomorphic bundles in $\mathscr{C}_s(E)$ admit unique Yang-Mills connections up to gauge equivalence. In other words, there is a homeomorphism 
$$
\mathscr{A}_s(P)/\mathscr{G}\longleftrightarrow\mathscr{C}_s(E)/\mathscr{G}
$$
\end{thmsub}
\begin{remksub}\normalfont
The original proof is more algebraic in flavor. A proof more in the spirit of the Atiyah-Bott paper was given by Donaldson in $[5]$. The spirit of this proof is carried on by the proof of Hermitian-Yang-Mills and the nonabelian Hodge theorem, which were both grew out of the developments from the Atiyah-Bott paper.
\end{remksub}



\section{Further research}
Too ideas of Atiyah and Bott's paper were extended to many directions. For example, Donaldson [6] used the space of solutions of the Yang-Mills equations to prove his celebrated theorem.

In another direction, work of Uhlenbeck and Yau [7] extended the study of slope stability to holomorphic vector bundles over higher dimensional Kähler manifolds, relating slop stability to the existence of Hermitian-Yang-Mills connections. Motivated by these ideas, Yau conjectured that the existence of Hermitian-Einstein connections on Fano manifolds would be related to another algebro-geometric notion of stability called $K$-stability. Recent work of Chen, Donaldson, and Sun [8] has mostly resolved this conjecture.

Another circle of ideas that grew out of the original paper have been the ideas around Higgs bundles. These were originally introduced by Hitchin [9], and have been used heavily in work by Simpson [10].

\section{Acknowledgement}
\begin{CJK}{UTF8}{gkai}
回顾大学四年至今,我成长了许多,从一个懵懂的少年,成长为了一个能够独立思考的人。在一路上,值得感谢的人有很多很多。

感谢那个在某一个深夜下定决心去追寻美的自己,即便最终一事无成也不后悔;感谢家人的理解与包容,虽然时有争吵,但我始终相信你们是我最坚实的后盾。

感谢在本科期间的各位老师,许多老师严谨的治学风范给我留下了非常深刻的印象,学为人师,行为世范;感谢徐泽老师开设的各种课程,让我免去了许多自学时的苦恼,徐老师扎实的基本功和认真严谨的治学态度让我十分敬佩;感谢我在清华的导师,不厌其烦地给我指引方向,在我心情低落时的诸多包容与鼓励,让我有继续前进的动力;感谢哈佛大学的陶布斯教授,感谢他如此高龄仍耐心地回复我的邮件,让我对他的教材的理解更加深刻;感谢魏子惟以及罗琪亮对我漏洞百出的论文的阅读,以及给我提出的若干意见。

感谢我在本科的诸多室友以及各种各样方式认识的朋友,诸位学弟学妹学长学姐,那些曾在我生命中出现又消失的人。你们给我枯燥的生活平添了诸多色彩,也让我更加开朗,更加相信自己。我始终坚信,每一段相遇都有它独特的意义,我所经历的一切,才塑造了当下饱满的我。

每一段故事都终将结束,而我将以这篇致谢告别我终将逝去的本科时光。剑尚未佩妥,转眼间已身处江湖,但勇敢的少年啊,快去创造新的奇迹。

\rightline{二零二二年五月三日}

\rightline{于北京大学 家园}
\end{CJK}
\begin{thebibliography}{99}
\bibitem{}
Atiyah, M. F., and R. Bott. “The Yang-Mills Equations over Riemann Surfaces.” Philosophical Transactions of the Royal Society of London. Series A, Mathematical and Physical Sciences, vol. 308, no. 1505, The Royal Society, 1983, pp. 523–615.
\bibitem{}
Taubes, C. (2011-10-13). Differential Geometry: Bundles, Connections, Metrics and Curvature. : Oxford University Press. Retrieved 4 Apr. 2022.
\bibitem{}
Jeffrey Jiang. Yang-Mills Minicourse notes. 2020. 
\bibitem{}
Gauge theory. https://empg.maths.ed.ac.uk/Activities/GT/
\bibitem{}
S. K. Donaldson. “A new proof of a theorem of Narasimhan and Seshadri”. In: J. Dif- ferential Geom. 18.2 (1983), pp. 269–277.
\bibitem{}
S. K. Donaldson. “An application of gauge theory to four-dimensional topology”. In: J. Differential Geom. 18.2 (1983), pp. 279–315. doi: 10.4310/jdg/1214437665. 
\bibitem{}
K. Uhlenbeck and S. T. Yau. “On the existence of hermitian-yang-mills connections
in stable vector bundles”. In: Communications on Pure and Applied Mathematics 39.S1 (1986), S257–S293. doi: 10.1002/cpa.3160390714.
\bibitem{}
Xiu-Xiong Chen, Simon Donaldson, and Song Sun. Kahler-Einstein metrics on Fano manifolds, I: approximation of metrics with cone singularities. 2012. arXiv: 1211.4566 [math.DG].
\bibitem{}
N. J. Hitchin. “The Self-Duality Equations on a Riemann Surface”. In: Proceedings of the London Mathematical Society s3-55.1 (July 1987), pp. 59–126. issn: 0024-6115. doi: 10.1112/plms/s3-55.1.59.
\bibitem{}
Carlos Simpson. “Higgs bundles and local systems”. en.
\bibitem{}
S. Kobayashi. Differential Geometry of Complex Vector Bundles. Princeton University Press, 1987.
\bibitem{}
S. Kobayashi and Nomizu K. Foundations of Differential Geometry Volume 1. Wiley Classics Library, 1996.
\bibitem{}
L. Tu. Differential Geometry. Springer, 2017.
\bibitem{}
D. McDuff and D. Salamon. Introduction to Symplectic Topology. Oxford Mathematical Monographs. Oxford University Press, 2017. isbn: 9780198794899.
\bibitem{}
D.S. Freed and K.K. Uhlenbeck. Instantons and Four-Manifolds. Mathematical Sci- ences Research Institute Berkeley, Calif: Mathematical Sciences Research Institute publications. Springer New York, 1991.
\end{thebibliography}


\newpage




\end{document}